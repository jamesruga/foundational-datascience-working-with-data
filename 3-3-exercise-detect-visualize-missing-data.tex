\documentclass[11pt]{article}

    \usepackage[breakable]{tcolorbox}
    \usepackage{parskip} % Stop auto-indenting (to mimic markdown behaviour)
    

    % Basic figure setup, for now with no caption control since it's done
    % automatically by Pandoc (which extracts ![](path) syntax from Markdown).
    \usepackage{graphicx}
    % Maintain compatibility with old templates. Remove in nbconvert 6.0
    \let\Oldincludegraphics\includegraphics
    % Ensure that by default, figures have no caption (until we provide a
    % proper Figure object with a Caption API and a way to capture that
    % in the conversion process - todo).
    \usepackage{caption}
    \DeclareCaptionFormat{nocaption}{}
    \captionsetup{format=nocaption,aboveskip=0pt,belowskip=0pt}

    \usepackage{float}
    \floatplacement{figure}{H} % forces figures to be placed at the correct location
    \usepackage{xcolor} % Allow colors to be defined
    \usepackage{enumerate} % Needed for markdown enumerations to work
    \usepackage{geometry} % Used to adjust the document margins
    \usepackage{amsmath} % Equations
    \usepackage{amssymb} % Equations
    \usepackage{textcomp} % defines textquotesingle
    % Hack from http://tex.stackexchange.com/a/47451/13684:
    \AtBeginDocument{%
        \def\PYZsq{\textquotesingle}% Upright quotes in Pygmentized code
    }
    \usepackage{upquote} % Upright quotes for verbatim code
    \usepackage{eurosym} % defines \euro

    \usepackage{iftex}
    \ifPDFTeX
        \usepackage[T1]{fontenc}
        \IfFileExists{alphabeta.sty}{
              \usepackage{alphabeta}
          }{
              \usepackage[mathletters]{ucs}
              \usepackage[utf8x]{inputenc}
          }
    \else
        \usepackage{fontspec}
        \usepackage{unicode-math}
    \fi

    \usepackage{fancyvrb} % verbatim replacement that allows latex
    \usepackage{grffile} % extends the file name processing of package graphics
                         % to support a larger range
    \makeatletter % fix for old versions of grffile with XeLaTeX
    \@ifpackagelater{grffile}{2019/11/01}
    {
      % Do nothing on new versions
    }
    {
      \def\Gread@@xetex#1{%
        \IfFileExists{"\Gin@base".bb}%
        {\Gread@eps{\Gin@base.bb}}%
        {\Gread@@xetex@aux#1}%
      }
    }
    \makeatother
    \usepackage[Export]{adjustbox} % Used to constrain images to a maximum size
    \adjustboxset{max size={0.9\linewidth}{0.9\paperheight}}

    % The hyperref package gives us a pdf with properly built
    % internal navigation ('pdf bookmarks' for the table of contents,
    % internal cross-reference links, web links for URLs, etc.)
    \usepackage{hyperref}
    % The default LaTeX title has an obnoxious amount of whitespace. By default,
    % titling removes some of it. It also provides customization options.
    \usepackage{titling}
    \usepackage{longtable} % longtable support required by pandoc >1.10
    \usepackage{booktabs}  % table support for pandoc > 1.12.2
    \usepackage{array}     % table support for pandoc >= 2.11.3
    \usepackage{calc}      % table minipage width calculation for pandoc >= 2.11.1
    \usepackage[inline]{enumitem} % IRkernel/repr support (it uses the enumerate* environment)
    \usepackage[normalem]{ulem} % ulem is needed to support strikethroughs (\sout)
                                % normalem makes italics be italics, not underlines
    \usepackage{mathrsfs}
    

    
    % Colors for the hyperref package
    \definecolor{urlcolor}{rgb}{0,.145,.698}
    \definecolor{linkcolor}{rgb}{.71,0.21,0.01}
    \definecolor{citecolor}{rgb}{.12,.54,.11}

    % ANSI colors
    \definecolor{ansi-black}{HTML}{3E424D}
    \definecolor{ansi-black-intense}{HTML}{282C36}
    \definecolor{ansi-red}{HTML}{E75C58}
    \definecolor{ansi-red-intense}{HTML}{B22B31}
    \definecolor{ansi-green}{HTML}{00A250}
    \definecolor{ansi-green-intense}{HTML}{007427}
    \definecolor{ansi-yellow}{HTML}{DDB62B}
    \definecolor{ansi-yellow-intense}{HTML}{B27D12}
    \definecolor{ansi-blue}{HTML}{208FFB}
    \definecolor{ansi-blue-intense}{HTML}{0065CA}
    \definecolor{ansi-magenta}{HTML}{D160C4}
    \definecolor{ansi-magenta-intense}{HTML}{A03196}
    \definecolor{ansi-cyan}{HTML}{60C6C8}
    \definecolor{ansi-cyan-intense}{HTML}{258F8F}
    \definecolor{ansi-white}{HTML}{C5C1B4}
    \definecolor{ansi-white-intense}{HTML}{A1A6B2}
    \definecolor{ansi-default-inverse-fg}{HTML}{FFFFFF}
    \definecolor{ansi-default-inverse-bg}{HTML}{000000}

    % common color for the border for error outputs.
    \definecolor{outerrorbackground}{HTML}{FFDFDF}

    % commands and environments needed by pandoc snippets
    % extracted from the output of `pandoc -s`
    \providecommand{\tightlist}{%
      \setlength{\itemsep}{0pt}\setlength{\parskip}{0pt}}
    \DefineVerbatimEnvironment{Highlighting}{Verbatim}{commandchars=\\\{\}}
    % Add ',fontsize=\small' for more characters per line
    \newenvironment{Shaded}{}{}
    \newcommand{\KeywordTok}[1]{\textcolor[rgb]{0.00,0.44,0.13}{\textbf{{#1}}}}
    \newcommand{\DataTypeTok}[1]{\textcolor[rgb]{0.56,0.13,0.00}{{#1}}}
    \newcommand{\DecValTok}[1]{\textcolor[rgb]{0.25,0.63,0.44}{{#1}}}
    \newcommand{\BaseNTok}[1]{\textcolor[rgb]{0.25,0.63,0.44}{{#1}}}
    \newcommand{\FloatTok}[1]{\textcolor[rgb]{0.25,0.63,0.44}{{#1}}}
    \newcommand{\CharTok}[1]{\textcolor[rgb]{0.25,0.44,0.63}{{#1}}}
    \newcommand{\StringTok}[1]{\textcolor[rgb]{0.25,0.44,0.63}{{#1}}}
    \newcommand{\CommentTok}[1]{\textcolor[rgb]{0.38,0.63,0.69}{\textit{{#1}}}}
    \newcommand{\OtherTok}[1]{\textcolor[rgb]{0.00,0.44,0.13}{{#1}}}
    \newcommand{\AlertTok}[1]{\textcolor[rgb]{1.00,0.00,0.00}{\textbf{{#1}}}}
    \newcommand{\FunctionTok}[1]{\textcolor[rgb]{0.02,0.16,0.49}{{#1}}}
    \newcommand{\RegionMarkerTok}[1]{{#1}}
    \newcommand{\ErrorTok}[1]{\textcolor[rgb]{1.00,0.00,0.00}{\textbf{{#1}}}}
    \newcommand{\NormalTok}[1]{{#1}}

    % Additional commands for more recent versions of Pandoc
    \newcommand{\ConstantTok}[1]{\textcolor[rgb]{0.53,0.00,0.00}{{#1}}}
    \newcommand{\SpecialCharTok}[1]{\textcolor[rgb]{0.25,0.44,0.63}{{#1}}}
    \newcommand{\VerbatimStringTok}[1]{\textcolor[rgb]{0.25,0.44,0.63}{{#1}}}
    \newcommand{\SpecialStringTok}[1]{\textcolor[rgb]{0.73,0.40,0.53}{{#1}}}
    \newcommand{\ImportTok}[1]{{#1}}
    \newcommand{\DocumentationTok}[1]{\textcolor[rgb]{0.73,0.13,0.13}{\textit{{#1}}}}
    \newcommand{\AnnotationTok}[1]{\textcolor[rgb]{0.38,0.63,0.69}{\textbf{\textit{{#1}}}}}
    \newcommand{\CommentVarTok}[1]{\textcolor[rgb]{0.38,0.63,0.69}{\textbf{\textit{{#1}}}}}
    \newcommand{\VariableTok}[1]{\textcolor[rgb]{0.10,0.09,0.49}{{#1}}}
    \newcommand{\ControlFlowTok}[1]{\textcolor[rgb]{0.00,0.44,0.13}{\textbf{{#1}}}}
    \newcommand{\OperatorTok}[1]{\textcolor[rgb]{0.40,0.40,0.40}{{#1}}}
    \newcommand{\BuiltInTok}[1]{{#1}}
    \newcommand{\ExtensionTok}[1]{{#1}}
    \newcommand{\PreprocessorTok}[1]{\textcolor[rgb]{0.74,0.48,0.00}{{#1}}}
    \newcommand{\AttributeTok}[1]{\textcolor[rgb]{0.49,0.56,0.16}{{#1}}}
    \newcommand{\InformationTok}[1]{\textcolor[rgb]{0.38,0.63,0.69}{\textbf{\textit{{#1}}}}}
    \newcommand{\WarningTok}[1]{\textcolor[rgb]{0.38,0.63,0.69}{\textbf{\textit{{#1}}}}}


    % Define a nice break command that doesn't care if a line doesn't already
    % exist.
    \def\br{\hspace*{\fill} \\* }
    % Math Jax compatibility definitions
    \def\gt{>}
    \def\lt{<}
    \let\Oldtex\TeX
    \let\Oldlatex\LaTeX
    \renewcommand{\TeX}{\textrm{\Oldtex}}
    \renewcommand{\LaTeX}{\textrm{\Oldlatex}}
    % Document parameters
    % Document title
    \title{Notebook}
    
    
    
    
    
    
    
% Pygments definitions
\makeatletter
\def\PY@reset{\let\PY@it=\relax \let\PY@bf=\relax%
    \let\PY@ul=\relax \let\PY@tc=\relax%
    \let\PY@bc=\relax \let\PY@ff=\relax}
\def\PY@tok#1{\csname PY@tok@#1\endcsname}
\def\PY@toks#1+{\ifx\relax#1\empty\else%
    \PY@tok{#1}\expandafter\PY@toks\fi}
\def\PY@do#1{\PY@bc{\PY@tc{\PY@ul{%
    \PY@it{\PY@bf{\PY@ff{#1}}}}}}}
\def\PY#1#2{\PY@reset\PY@toks#1+\relax+\PY@do{#2}}

\@namedef{PY@tok@w}{\def\PY@tc##1{\textcolor[rgb]{0.73,0.73,0.73}{##1}}}
\@namedef{PY@tok@c}{\let\PY@it=\textit\def\PY@tc##1{\textcolor[rgb]{0.24,0.48,0.48}{##1}}}
\@namedef{PY@tok@cp}{\def\PY@tc##1{\textcolor[rgb]{0.61,0.40,0.00}{##1}}}
\@namedef{PY@tok@k}{\let\PY@bf=\textbf\def\PY@tc##1{\textcolor[rgb]{0.00,0.50,0.00}{##1}}}
\@namedef{PY@tok@kp}{\def\PY@tc##1{\textcolor[rgb]{0.00,0.50,0.00}{##1}}}
\@namedef{PY@tok@kt}{\def\PY@tc##1{\textcolor[rgb]{0.69,0.00,0.25}{##1}}}
\@namedef{PY@tok@o}{\def\PY@tc##1{\textcolor[rgb]{0.40,0.40,0.40}{##1}}}
\@namedef{PY@tok@ow}{\let\PY@bf=\textbf\def\PY@tc##1{\textcolor[rgb]{0.67,0.13,1.00}{##1}}}
\@namedef{PY@tok@nb}{\def\PY@tc##1{\textcolor[rgb]{0.00,0.50,0.00}{##1}}}
\@namedef{PY@tok@nf}{\def\PY@tc##1{\textcolor[rgb]{0.00,0.00,1.00}{##1}}}
\@namedef{PY@tok@nc}{\let\PY@bf=\textbf\def\PY@tc##1{\textcolor[rgb]{0.00,0.00,1.00}{##1}}}
\@namedef{PY@tok@nn}{\let\PY@bf=\textbf\def\PY@tc##1{\textcolor[rgb]{0.00,0.00,1.00}{##1}}}
\@namedef{PY@tok@ne}{\let\PY@bf=\textbf\def\PY@tc##1{\textcolor[rgb]{0.80,0.25,0.22}{##1}}}
\@namedef{PY@tok@nv}{\def\PY@tc##1{\textcolor[rgb]{0.10,0.09,0.49}{##1}}}
\@namedef{PY@tok@no}{\def\PY@tc##1{\textcolor[rgb]{0.53,0.00,0.00}{##1}}}
\@namedef{PY@tok@nl}{\def\PY@tc##1{\textcolor[rgb]{0.46,0.46,0.00}{##1}}}
\@namedef{PY@tok@ni}{\let\PY@bf=\textbf\def\PY@tc##1{\textcolor[rgb]{0.44,0.44,0.44}{##1}}}
\@namedef{PY@tok@na}{\def\PY@tc##1{\textcolor[rgb]{0.41,0.47,0.13}{##1}}}
\@namedef{PY@tok@nt}{\let\PY@bf=\textbf\def\PY@tc##1{\textcolor[rgb]{0.00,0.50,0.00}{##1}}}
\@namedef{PY@tok@nd}{\def\PY@tc##1{\textcolor[rgb]{0.67,0.13,1.00}{##1}}}
\@namedef{PY@tok@s}{\def\PY@tc##1{\textcolor[rgb]{0.73,0.13,0.13}{##1}}}
\@namedef{PY@tok@sd}{\let\PY@it=\textit\def\PY@tc##1{\textcolor[rgb]{0.73,0.13,0.13}{##1}}}
\@namedef{PY@tok@si}{\let\PY@bf=\textbf\def\PY@tc##1{\textcolor[rgb]{0.64,0.35,0.47}{##1}}}
\@namedef{PY@tok@se}{\let\PY@bf=\textbf\def\PY@tc##1{\textcolor[rgb]{0.67,0.36,0.12}{##1}}}
\@namedef{PY@tok@sr}{\def\PY@tc##1{\textcolor[rgb]{0.64,0.35,0.47}{##1}}}
\@namedef{PY@tok@ss}{\def\PY@tc##1{\textcolor[rgb]{0.10,0.09,0.49}{##1}}}
\@namedef{PY@tok@sx}{\def\PY@tc##1{\textcolor[rgb]{0.00,0.50,0.00}{##1}}}
\@namedef{PY@tok@m}{\def\PY@tc##1{\textcolor[rgb]{0.40,0.40,0.40}{##1}}}
\@namedef{PY@tok@gh}{\let\PY@bf=\textbf\def\PY@tc##1{\textcolor[rgb]{0.00,0.00,0.50}{##1}}}
\@namedef{PY@tok@gu}{\let\PY@bf=\textbf\def\PY@tc##1{\textcolor[rgb]{0.50,0.00,0.50}{##1}}}
\@namedef{PY@tok@gd}{\def\PY@tc##1{\textcolor[rgb]{0.63,0.00,0.00}{##1}}}
\@namedef{PY@tok@gi}{\def\PY@tc##1{\textcolor[rgb]{0.00,0.52,0.00}{##1}}}
\@namedef{PY@tok@gr}{\def\PY@tc##1{\textcolor[rgb]{0.89,0.00,0.00}{##1}}}
\@namedef{PY@tok@ge}{\let\PY@it=\textit}
\@namedef{PY@tok@gs}{\let\PY@bf=\textbf}
\@namedef{PY@tok@gp}{\let\PY@bf=\textbf\def\PY@tc##1{\textcolor[rgb]{0.00,0.00,0.50}{##1}}}
\@namedef{PY@tok@go}{\def\PY@tc##1{\textcolor[rgb]{0.44,0.44,0.44}{##1}}}
\@namedef{PY@tok@gt}{\def\PY@tc##1{\textcolor[rgb]{0.00,0.27,0.87}{##1}}}
\@namedef{PY@tok@err}{\def\PY@bc##1{{\setlength{\fboxsep}{\string -\fboxrule}\fcolorbox[rgb]{1.00,0.00,0.00}{1,1,1}{\strut ##1}}}}
\@namedef{PY@tok@kc}{\let\PY@bf=\textbf\def\PY@tc##1{\textcolor[rgb]{0.00,0.50,0.00}{##1}}}
\@namedef{PY@tok@kd}{\let\PY@bf=\textbf\def\PY@tc##1{\textcolor[rgb]{0.00,0.50,0.00}{##1}}}
\@namedef{PY@tok@kn}{\let\PY@bf=\textbf\def\PY@tc##1{\textcolor[rgb]{0.00,0.50,0.00}{##1}}}
\@namedef{PY@tok@kr}{\let\PY@bf=\textbf\def\PY@tc##1{\textcolor[rgb]{0.00,0.50,0.00}{##1}}}
\@namedef{PY@tok@bp}{\def\PY@tc##1{\textcolor[rgb]{0.00,0.50,0.00}{##1}}}
\@namedef{PY@tok@fm}{\def\PY@tc##1{\textcolor[rgb]{0.00,0.00,1.00}{##1}}}
\@namedef{PY@tok@vc}{\def\PY@tc##1{\textcolor[rgb]{0.10,0.09,0.49}{##1}}}
\@namedef{PY@tok@vg}{\def\PY@tc##1{\textcolor[rgb]{0.10,0.09,0.49}{##1}}}
\@namedef{PY@tok@vi}{\def\PY@tc##1{\textcolor[rgb]{0.10,0.09,0.49}{##1}}}
\@namedef{PY@tok@vm}{\def\PY@tc##1{\textcolor[rgb]{0.10,0.09,0.49}{##1}}}
\@namedef{PY@tok@sa}{\def\PY@tc##1{\textcolor[rgb]{0.73,0.13,0.13}{##1}}}
\@namedef{PY@tok@sb}{\def\PY@tc##1{\textcolor[rgb]{0.73,0.13,0.13}{##1}}}
\@namedef{PY@tok@sc}{\def\PY@tc##1{\textcolor[rgb]{0.73,0.13,0.13}{##1}}}
\@namedef{PY@tok@dl}{\def\PY@tc##1{\textcolor[rgb]{0.73,0.13,0.13}{##1}}}
\@namedef{PY@tok@s2}{\def\PY@tc##1{\textcolor[rgb]{0.73,0.13,0.13}{##1}}}
\@namedef{PY@tok@sh}{\def\PY@tc##1{\textcolor[rgb]{0.73,0.13,0.13}{##1}}}
\@namedef{PY@tok@s1}{\def\PY@tc##1{\textcolor[rgb]{0.73,0.13,0.13}{##1}}}
\@namedef{PY@tok@mb}{\def\PY@tc##1{\textcolor[rgb]{0.40,0.40,0.40}{##1}}}
\@namedef{PY@tok@mf}{\def\PY@tc##1{\textcolor[rgb]{0.40,0.40,0.40}{##1}}}
\@namedef{PY@tok@mh}{\def\PY@tc##1{\textcolor[rgb]{0.40,0.40,0.40}{##1}}}
\@namedef{PY@tok@mi}{\def\PY@tc##1{\textcolor[rgb]{0.40,0.40,0.40}{##1}}}
\@namedef{PY@tok@il}{\def\PY@tc##1{\textcolor[rgb]{0.40,0.40,0.40}{##1}}}
\@namedef{PY@tok@mo}{\def\PY@tc##1{\textcolor[rgb]{0.40,0.40,0.40}{##1}}}
\@namedef{PY@tok@ch}{\let\PY@it=\textit\def\PY@tc##1{\textcolor[rgb]{0.24,0.48,0.48}{##1}}}
\@namedef{PY@tok@cm}{\let\PY@it=\textit\def\PY@tc##1{\textcolor[rgb]{0.24,0.48,0.48}{##1}}}
\@namedef{PY@tok@cpf}{\let\PY@it=\textit\def\PY@tc##1{\textcolor[rgb]{0.24,0.48,0.48}{##1}}}
\@namedef{PY@tok@c1}{\let\PY@it=\textit\def\PY@tc##1{\textcolor[rgb]{0.24,0.48,0.48}{##1}}}
\@namedef{PY@tok@cs}{\let\PY@it=\textit\def\PY@tc##1{\textcolor[rgb]{0.24,0.48,0.48}{##1}}}

\def\PYZbs{\char`\\}
\def\PYZus{\char`\_}
\def\PYZob{\char`\{}
\def\PYZcb{\char`\}}
\def\PYZca{\char`\^}
\def\PYZam{\char`\&}
\def\PYZlt{\char`\<}
\def\PYZgt{\char`\>}
\def\PYZsh{\char`\#}
\def\PYZpc{\char`\%}
\def\PYZdl{\char`\$}
\def\PYZhy{\char`\-}
\def\PYZsq{\char`\'}
\def\PYZdq{\char`\"}
\def\PYZti{\char`\~}
% for compatibility with earlier versions
\def\PYZat{@}
\def\PYZlb{[}
\def\PYZrb{]}
\makeatother


    % For linebreaks inside Verbatim environment from package fancyvrb.
    \makeatletter
        \newbox\Wrappedcontinuationbox
        \newbox\Wrappedvisiblespacebox
        \newcommand*\Wrappedvisiblespace {\textcolor{red}{\textvisiblespace}}
        \newcommand*\Wrappedcontinuationsymbol {\textcolor{red}{\llap{\tiny$\m@th\hookrightarrow$}}}
        \newcommand*\Wrappedcontinuationindent {3ex }
        \newcommand*\Wrappedafterbreak {\kern\Wrappedcontinuationindent\copy\Wrappedcontinuationbox}
        % Take advantage of the already applied Pygments mark-up to insert
        % potential linebreaks for TeX processing.
        %        {, <, #, %, $, ' and ": go to next line.
        %        _, }, ^, &, >, - and ~: stay at end of broken line.
        % Use of \textquotesingle for straight quote.
        \newcommand*\Wrappedbreaksatspecials {%
            \def\PYGZus{\discretionary{\char`\_}{\Wrappedafterbreak}{\char`\_}}%
            \def\PYGZob{\discretionary{}{\Wrappedafterbreak\char`\{}{\char`\{}}%
            \def\PYGZcb{\discretionary{\char`\}}{\Wrappedafterbreak}{\char`\}}}%
            \def\PYGZca{\discretionary{\char`\^}{\Wrappedafterbreak}{\char`\^}}%
            \def\PYGZam{\discretionary{\char`\&}{\Wrappedafterbreak}{\char`\&}}%
            \def\PYGZlt{\discretionary{}{\Wrappedafterbreak\char`\<}{\char`\<}}%
            \def\PYGZgt{\discretionary{\char`\>}{\Wrappedafterbreak}{\char`\>}}%
            \def\PYGZsh{\discretionary{}{\Wrappedafterbreak\char`\#}{\char`\#}}%
            \def\PYGZpc{\discretionary{}{\Wrappedafterbreak\char`\%}{\char`\%}}%
            \def\PYGZdl{\discretionary{}{\Wrappedafterbreak\char`\$}{\char`\$}}%
            \def\PYGZhy{\discretionary{\char`\-}{\Wrappedafterbreak}{\char`\-}}%
            \def\PYGZsq{\discretionary{}{\Wrappedafterbreak\textquotesingle}{\textquotesingle}}%
            \def\PYGZdq{\discretionary{}{\Wrappedafterbreak\char`\"}{\char`\"}}%
            \def\PYGZti{\discretionary{\char`\~}{\Wrappedafterbreak}{\char`\~}}%
        }
        % Some characters . , ; ? ! / are not pygmentized.
        % This macro makes them "active" and they will insert potential linebreaks
        \newcommand*\Wrappedbreaksatpunct {%
            \lccode`\~`\.\lowercase{\def~}{\discretionary{\hbox{\char`\.}}{\Wrappedafterbreak}{\hbox{\char`\.}}}%
            \lccode`\~`\,\lowercase{\def~}{\discretionary{\hbox{\char`\,}}{\Wrappedafterbreak}{\hbox{\char`\,}}}%
            \lccode`\~`\;\lowercase{\def~}{\discretionary{\hbox{\char`\;}}{\Wrappedafterbreak}{\hbox{\char`\;}}}%
            \lccode`\~`\:\lowercase{\def~}{\discretionary{\hbox{\char`\:}}{\Wrappedafterbreak}{\hbox{\char`\:}}}%
            \lccode`\~`\?\lowercase{\def~}{\discretionary{\hbox{\char`\?}}{\Wrappedafterbreak}{\hbox{\char`\?}}}%
            \lccode`\~`\!\lowercase{\def~}{\discretionary{\hbox{\char`\!}}{\Wrappedafterbreak}{\hbox{\char`\!}}}%
            \lccode`\~`\/\lowercase{\def~}{\discretionary{\hbox{\char`\/}}{\Wrappedafterbreak}{\hbox{\char`\/}}}%
            \catcode`\.\active
            \catcode`\,\active
            \catcode`\;\active
            \catcode`\:\active
            \catcode`\?\active
            \catcode`\!\active
            \catcode`\/\active
            \lccode`\~`\~
        }
    \makeatother

    \let\OriginalVerbatim=\Verbatim
    \makeatletter
    \renewcommand{\Verbatim}[1][1]{%
        %\parskip\z@skip
        \sbox\Wrappedcontinuationbox {\Wrappedcontinuationsymbol}%
        \sbox\Wrappedvisiblespacebox {\FV@SetupFont\Wrappedvisiblespace}%
        \def\FancyVerbFormatLine ##1{\hsize\linewidth
            \vtop{\raggedright\hyphenpenalty\z@\exhyphenpenalty\z@
                \doublehyphendemerits\z@\finalhyphendemerits\z@
                \strut ##1\strut}%
        }%
        % If the linebreak is at a space, the latter will be displayed as visible
        % space at end of first line, and a continuation symbol starts next line.
        % Stretch/shrink are however usually zero for typewriter font.
        \def\FV@Space {%
            \nobreak\hskip\z@ plus\fontdimen3\font minus\fontdimen4\font
            \discretionary{\copy\Wrappedvisiblespacebox}{\Wrappedafterbreak}
            {\kern\fontdimen2\font}%
        }%

        % Allow breaks at special characters using \PYG... macros.
        \Wrappedbreaksatspecials
        % Breaks at punctuation characters . , ; ? ! and / need catcode=\active
        \OriginalVerbatim[#1,codes*=\Wrappedbreaksatpunct]%
    }
    \makeatother

    % Exact colors from NB
    \definecolor{incolor}{HTML}{303F9F}
    \definecolor{outcolor}{HTML}{D84315}
    \definecolor{cellborder}{HTML}{CFCFCF}
    \definecolor{cellbackground}{HTML}{F7F7F7}

    % prompt
    \makeatletter
    \newcommand{\boxspacing}{\kern\kvtcb@left@rule\kern\kvtcb@boxsep}
    \makeatother
    \newcommand{\prompt}[4]{
        {\ttfamily\llap{{\color{#2}[#3]:\hspace{3pt}#4}}\vspace{-\baselineskip}}
    }
    

    
    % Prevent overflowing lines due to hard-to-break entities
    \sloppy
    % Setup hyperref package
    \hypersetup{
      breaklinks=true,  % so long urls are correctly broken across lines
      colorlinks=true,
      urlcolor=urlcolor,
      linkcolor=linkcolor,
      citecolor=citecolor,
      }
    % Slightly bigger margins than the latex defaults
    
    \geometry{verbose,tmargin=1in,bmargin=1in,lmargin=1in,rmargin=1in}
    
    

\begin{document}
    
    \maketitle
    
    

    
    \hypertarget{exercise-titanic-dataset---find-and-visualize-missing-data}{%
\section{Exercise: Titanic Dataset - Find and Visualize Missing
Data}\label{exercise-titanic-dataset---find-and-visualize-missing-data}}

Datasets can often have missing data, which can cause problems when we
perform machine learning. Missing data can be hard to spot at a first
glance.

In our scenario, we obtained a list of passengers on the failed maiden
voyage of the Titanic. We'd like to know which factors predicted
passenger survival. For our first task, which we'll perform here, we'll
check whether our dataset has missing information.

    \hypertarget{the-required-graphing-library-package}{%
\subsection{The required `graphing' library
package}\label{the-required-graphing-library-package}}

This tutorial relies on the `graphing' library package resource.
Depending on the environment you use to execute the code in this
notebook, you might need to install that package to proceed. To install
the `graphing' package, uncomment and execute this notebook cell code:

    \begin{tcolorbox}[breakable, size=fbox, boxrule=1pt, pad at break*=1mm,colback=cellbackground, colframe=cellborder]
\prompt{In}{incolor}{14}{\boxspacing}
\begin{Verbatim}[commandchars=\\\{\}]
\PY{c+c1}{\PYZsh{} \PYZpc{}pip install graphing}
\end{Verbatim}
\end{tcolorbox}

    \hypertarget{preparing-data}{%
\subsection{Preparing data}\label{preparing-data}}

Let's use Pandas to load the dataset and take a cursory look at it:

    \begin{tcolorbox}[breakable, size=fbox, boxrule=1pt, pad at break*=1mm,colback=cellbackground, colframe=cellborder]
\prompt{In}{incolor}{15}{\boxspacing}
\begin{Verbatim}[commandchars=\\\{\}]
\PY{k+kn}{import} \PY{n+nn}{pandas} \PY{k}{as} \PY{n+nn}{pd}
\PY{o}{!}pip\PY{+w}{ }install\PY{+w}{ }missingno

\PY{c+c1}{\PYZsh{} Load data from our dataset file into a pandas dataframe}
\PY{o}{!}wget\PY{+w}{ }https://raw.githubusercontent.com/MicrosoftDocs/mslearn\PYZhy{}introduction\PYZhy{}to\PYZhy{}machine\PYZhy{}learning/main/Data/titanic.csv
\PY{o}{!}wget\PY{+w}{ }https://raw.githubusercontent.com/MicrosoftDocs/mslearn\PYZhy{}introduction\PYZhy{}to\PYZhy{}machine\PYZhy{}learning/main/graphing.py
\PY{n}{dataset} \PY{o}{=} \PY{n}{pd}\PY{o}{.}\PY{n}{read\PYZus{}csv}\PY{p}{(}\PY{l+s+s1}{\PYZsq{}}\PY{l+s+s1}{titanic.csv}\PY{l+s+s1}{\PYZsq{}}\PY{p}{,} \PY{n}{index\PYZus{}col}\PY{o}{=}\PY{k+kc}{False}\PY{p}{,} \PY{n}{sep}\PY{o}{=}\PY{l+s+s2}{\PYZdq{}}\PY{l+s+s2}{,}\PY{l+s+s2}{\PYZdq{}}\PY{p}{,} \PY{n}{header}\PY{o}{=}\PY{l+m+mi}{0}\PY{p}{)}

\PY{c+c1}{\PYZsh{} Let\PYZsq{}s take a look at the data}
\PY{n}{dataset}\PY{o}{.}\PY{n}{head}\PY{p}{(}\PY{p}{)}
\end{Verbatim}
\end{tcolorbox}

    \begin{Verbatim}[commandchars=\\\{\}]
Requirement already satisfied: missingno in
/anaconda/envs/azureml\_py38/lib/python3.8/site-packages (0.5.1)
Requirement already satisfied: numpy in
/anaconda/envs/azureml\_py38/lib/python3.8/site-packages (from missingno)
(1.21.6)
Requirement already satisfied: matplotlib in
/anaconda/envs/azureml\_py38/lib/python3.8/site-packages (from missingno) (3.2.1)
Requirement already satisfied: scipy in
/anaconda/envs/azureml\_py38/lib/python3.8/site-packages (from missingno) (1.5.3)
Requirement already satisfied: seaborn in
/anaconda/envs/azureml\_py38/lib/python3.8/site-packages (from missingno)
(0.12.0)
Requirement already satisfied: python-dateutil>=2.1 in
/anaconda/envs/azureml\_py38/lib/python3.8/site-packages (from
matplotlib->missingno) (2.8.2)
Requirement already satisfied: pyparsing!=2.0.4,!=2.1.2,!=2.1.6,>=2.0.1 in
/anaconda/envs/azureml\_py38/lib/python3.8/site-packages (from
matplotlib->missingno) (3.0.9)
Requirement already satisfied: kiwisolver>=1.0.1 in
/anaconda/envs/azureml\_py38/lib/python3.8/site-packages (from
matplotlib->missingno) (1.4.4)
Requirement already satisfied: cycler>=0.10 in
/anaconda/envs/azureml\_py38/lib/python3.8/site-packages (from
matplotlib->missingno) (0.11.0)
Requirement already satisfied: pandas>=0.25 in
/anaconda/envs/azureml\_py38/lib/python3.8/site-packages (from
seaborn->missingno) (1.1.5)
Requirement already satisfied: six>=1.5 in
/anaconda/envs/azureml\_py38/lib/python3.8/site-packages (from python-
dateutil>=2.1->matplotlib->missingno) (1.16.0)
Requirement already satisfied: pytz>=2017.2 in
/anaconda/envs/azureml\_py38/lib/python3.8/site-packages (from
pandas>=0.25->seaborn->missingno) (2022.1)
--2023-08-19 03:02:59--
https://raw.githubusercontent.com/MicrosoftDocs/mslearn-introduction-to-machine-
learning/main/Data/titanic.csv
Resolving raw.githubusercontent.com (raw.githubusercontent.com){\ldots}
185.199.108.133, 185.199.109.133, 185.199.110.133, {\ldots}
Connecting to raw.githubusercontent.com
(raw.githubusercontent.com)|185.199.108.133|:443{\ldots} connected.
HTTP request sent, awaiting response{\ldots} 200 OK
Length: 61194 (60K) [text/plain]
Saving to: ‘titanic.csv.1’

titanic.csv.1       100\%[===================>]  59.76K  --.-KB/s    in 0.001s

2023-08-19 03:02:59 (107 MB/s) - ‘titanic.csv.1’ saved [61194/61194]

--2023-08-19 03:03:00--
https://raw.githubusercontent.com/MicrosoftDocs/mslearn-introduction-to-machine-
learning/main/graphing.py
Resolving raw.githubusercontent.com (raw.githubusercontent.com){\ldots}
185.199.111.133, 185.199.110.133, 185.199.109.133, {\ldots}
Connecting to raw.githubusercontent.com
(raw.githubusercontent.com)|185.199.111.133|:443{\ldots} connected.
HTTP request sent, awaiting response{\ldots} 200 OK
Length: 21511 (21K) [text/plain]
Saving to: ‘graphing.py.1’

graphing.py.1       100\%[===================>]  21.01K  --.-KB/s    in 0s

2023-08-19 03:03:00 (72.6 MB/s) - ‘graphing.py.1’ saved [21511/21511]

    \end{Verbatim}

            \begin{tcolorbox}[breakable, size=fbox, boxrule=.5pt, pad at break*=1mm, opacityfill=0]
\prompt{Out}{outcolor}{15}{\boxspacing}
\begin{Verbatim}[commandchars=\\\{\}]
   PassengerId  Survived  Pclass  \textbackslash{}
0            1         0       3
1            2         1       1
2            3         1       3
3            4         1       1
4            5         0       3

                                                Name     Sex   Age  SibSp  \textbackslash{}
0                            Braund, Mr. Owen Harris    male  22.0      1
1  Cumings, Mrs. John Bradley (Florence Briggs Th{\ldots}  female  38.0      1
2                             Heikkinen, Miss. Laina  female  26.0      0
3       Futrelle, Mrs. Jacques Heath (Lily May Peel)  female  35.0      1
4                           Allen, Mr. William Henry    male  35.0      0

   Parch            Ticket     Fare Cabin Embarked
0      0         A/5 21171   7.2500   NaN        S
1      0          PC 17599  71.2833   C85        C
2      0  STON/O2. 3101282   7.9250   NaN        S
3      0            113803  53.1000  C123        S
4      0            373450   8.0500   NaN        S
\end{Verbatim}
\end{tcolorbox}
        
    Now, we'll see how many samples and columns we have:

    \begin{tcolorbox}[breakable, size=fbox, boxrule=1pt, pad at break*=1mm,colback=cellbackground, colframe=cellborder]
\prompt{In}{incolor}{16}{\boxspacing}
\begin{Verbatim}[commandchars=\\\{\}]
\PY{c+c1}{\PYZsh{} Shape tells us how many rows and columns we have}
\PY{n+nb}{print}\PY{p}{(}\PY{n}{dataset}\PY{o}{.}\PY{n}{shape}\PY{p}{)}
\end{Verbatim}
\end{tcolorbox}

    \begin{Verbatim}[commandchars=\\\{\}]
(891, 12)
    \end{Verbatim}

    We have data for 891 passengers, each described by 12 different
variables.

\hypertarget{finding-missing-data}{%
\subsection{Finding Missing Data}\label{finding-missing-data}}

Do we have a complete dataset?

No.~We know from history that there were more than 2000 people on the
Titanic, so we know straight away that we are missing information on
more than 1000 people!

How can we tell if the data we have available is complete?

We could print the entire dataset, but this could involve human error,
and it would become impractical with this many samples.

A better option would use \texttt{pandas} to report the columns that
have ``empty'' cells:

    \begin{tcolorbox}[breakable, size=fbox, boxrule=1pt, pad at break*=1mm,colback=cellbackground, colframe=cellborder]
\prompt{In}{incolor}{17}{\boxspacing}
\begin{Verbatim}[commandchars=\\\{\}]
\PY{c+c1}{\PYZsh{} Calculate the number of empty cells in each column}
\PY{c+c1}{\PYZsh{} The following line consists of three commands. Try}
\PY{c+c1}{\PYZsh{} to think about how they work together to calculate}
\PY{c+c1}{\PYZsh{} the number of missing entries per column}
\PY{n}{missing\PYZus{}data} \PY{o}{=} \PY{n}{dataset}\PY{o}{.}\PY{n}{isnull}\PY{p}{(}\PY{p}{)}\PY{o}{.}\PY{n}{sum}\PY{p}{(}\PY{p}{)}\PY{o}{.}\PY{n}{to\PYZus{}frame}\PY{p}{(}\PY{p}{)}

\PY{c+c1}{\PYZsh{} Rename column holding the sums}
\PY{n}{missing\PYZus{}data} \PY{o}{=} \PY{n}{missing\PYZus{}data}\PY{o}{.}\PY{n}{rename}\PY{p}{(}\PY{n}{columns}\PY{o}{=}\PY{p}{\PYZob{}}\PY{l+m+mi}{0}\PY{p}{:}\PY{l+s+s1}{\PYZsq{}}\PY{l+s+s1}{Empty Cells}\PY{l+s+s1}{\PYZsq{}}\PY{p}{\PYZcb{}}\PY{p}{)}

\PY{c+c1}{\PYZsh{} Print the results}
\PY{n+nb}{print}\PY{p}{(}\PY{n}{missing\PYZus{}data}\PY{p}{)}
\end{Verbatim}
\end{tcolorbox}

    \begin{Verbatim}[commandchars=\\\{\}]
             Empty Cells
PassengerId            0
Survived               0
Pclass                 0
Name                   0
Sex                    0
Age                  177
SibSp                  0
Parch                  0
Ticket                 0
Fare                   0
Cabin                687
Embarked               2
    \end{Verbatim}

    It looks like we don't know the age of 177 passengers, and we don't know
if two of them even embarked.

Cabin information for a whopping 687 persons is also missing.

\hypertarget{missing-data-visualizations}{%
\subsection{Missing Data
Visualizations}\label{missing-data-visualizations}}

Sometimes it can help if we can see if the missing data form some kind
of pattern.

We can plot the absence of data in a few ways. One of the most helpful
is to literally plot gaps in the dataset:

    \begin{tcolorbox}[breakable, size=fbox, boxrule=1pt, pad at break*=1mm,colback=cellbackground, colframe=cellborder]
\prompt{In}{incolor}{18}{\boxspacing}
\begin{Verbatim}[commandchars=\\\{\}]
\PY{c+c1}{\PYZsh{} import missingno package}
\PY{k+kn}{import} \PY{n+nn}{missingno} \PY{k}{as} \PY{n+nn}{msno}

\PY{c+c1}{\PYZsh{} Plot a matrix chart, set chart and font size}
\PY{n}{msno}\PY{o}{.}\PY{n}{matrix}\PY{p}{(}\PY{n}{dataset}\PY{p}{,} \PY{n}{figsize}\PY{o}{=}\PY{p}{(}\PY{l+m+mi}{10}\PY{p}{,}\PY{l+m+mi}{5}\PY{p}{)}\PY{p}{,} \PY{n}{fontsize}\PY{o}{=}\PY{l+m+mi}{11}\PY{p}{)}
\end{Verbatim}
\end{tcolorbox}

            \begin{tcolorbox}[breakable, size=fbox, boxrule=.5pt, pad at break*=1mm, opacityfill=0]
\prompt{Out}{outcolor}{18}{\boxspacing}
\begin{Verbatim}[commandchars=\\\{\}]
<matplotlib.axes.\_subplots.AxesSubplot at 0x7f50c7121a30>
\end{Verbatim}
\end{tcolorbox}
        
    \begin{center}
    \adjustimage{max size={0.9\linewidth}{0.9\paperheight}}{output_10_1.png}
    \end{center}
    { \hspace*{\fill} \\}
    
    The white bars in the graph show missing data. Here, the patterns aren't
visually clear, but maybe many passengers with missing \texttt{Age}
information are also missing \texttt{Cabin} information.

\hypertarget{identifying-individual-passengers-with-missing-information.}{%
\subsection{Identifying Individual Passengers with Missing
Information.}\label{identifying-individual-passengers-with-missing-information.}}

Let's use pandas to get a list of passengers of unknown age:

    \begin{tcolorbox}[breakable, size=fbox, boxrule=1pt, pad at break*=1mm,colback=cellbackground, colframe=cellborder]
\prompt{In}{incolor}{19}{\boxspacing}
\begin{Verbatim}[commandchars=\\\{\}]
\PY{c+c1}{\PYZsh{} Select Passengers with unknown age}
\PY{c+c1}{\PYZsh{} Notice how we use .isnull() rows with no value}
\PY{n}{unknown\PYZus{}age} \PY{o}{=} \PY{n}{dataset}\PY{p}{[}\PY{n}{dataset}\PY{p}{[}\PY{l+s+s2}{\PYZdq{}}\PY{l+s+s2}{Age}\PY{l+s+s2}{\PYZdq{}}\PY{p}{]}\PY{o}{.}\PY{n}{isnull}\PY{p}{(}\PY{p}{)}\PY{p}{]}

\PY{c+c1}{\PYZsh{} Print only the columns we want for the moment (to better fit the screen)}
\PY{c+c1}{\PYZsh{} limit output to 20 rows}
\PY{n}{unknown\PYZus{}age}\PY{p}{[}\PY{p}{[}\PY{l+s+s2}{\PYZdq{}}\PY{l+s+s2}{PassengerId}\PY{l+s+s2}{\PYZdq{}}\PY{p}{,}\PY{l+s+s2}{\PYZdq{}}\PY{l+s+s2}{Name}\PY{l+s+s2}{\PYZdq{}}\PY{p}{,} \PY{l+s+s2}{\PYZdq{}}\PY{l+s+s2}{Survived}\PY{l+s+s2}{\PYZdq{}}\PY{p}{,} \PY{l+s+s2}{\PYZdq{}}\PY{l+s+s2}{Age}\PY{l+s+s2}{\PYZdq{}}\PY{p}{]}\PY{p}{]}\PY{p}{[}\PY{p}{:}\PY{l+m+mi}{20}\PY{p}{]}
\end{Verbatim}
\end{tcolorbox}

            \begin{tcolorbox}[breakable, size=fbox, boxrule=.5pt, pad at break*=1mm, opacityfill=0]
\prompt{Out}{outcolor}{19}{\boxspacing}
\begin{Verbatim}[commandchars=\\\{\}]
    PassengerId                                            Name  Survived  Age
5             6                                Moran, Mr. James         0  NaN
17           18                    Williams, Mr. Charles Eugene         1  NaN
19           20                         Masselmani, Mrs. Fatima         1  NaN
26           27                         Emir, Mr. Farred Chehab         0  NaN
28           29                   O'Dwyer, Miss. Ellen "Nellie"         1  NaN
29           30                             Todoroff, Mr. Lalio         0  NaN
31           32  Spencer, Mrs. William Augustus (Marie Eugenie)         1  NaN
32           33                        Glynn, Miss. Mary Agatha         1  NaN
36           37                                Mamee, Mr. Hanna         1  NaN
42           43                             Kraeff, Mr. Theodor         0  NaN
45           46                        Rogers, Mr. William John         0  NaN
46           47                               Lennon, Mr. Denis         0  NaN
47           48                       O'Driscoll, Miss. Bridget         1  NaN
48           49                             Samaan, Mr. Youssef         0  NaN
55           56                               Woolner, Mr. Hugh         1  NaN
64           65                           Stewart, Mr. Albert A         0  NaN
65           66                        Moubarek, Master. Gerios         1  NaN
76           77                               Staneff, Mr. Ivan         0  NaN
77           78                        Moutal, Mr. Rahamin Haim         0  NaN
82           83                  McDermott, Miss. Brigdet Delia         1  NaN
\end{Verbatim}
\end{tcolorbox}
        
    This technique lists the passengers with missing \texttt{Cabin} or
\texttt{Embarked} information as well. Let's combine these using an
\texttt{AND}, to see how many passengers are missing both Cabin and Age
information

    \begin{tcolorbox}[breakable, size=fbox, boxrule=1pt, pad at break*=1mm,colback=cellbackground, colframe=cellborder]
\prompt{In}{incolor}{20}{\boxspacing}
\begin{Verbatim}[commandchars=\\\{\}]
\PY{c+c1}{\PYZsh{} Find those passengers with missing age or cabin information}
\PY{n}{missing\PYZus{}age} \PY{o}{=} \PY{n}{dataset}\PY{p}{[}\PY{l+s+s2}{\PYZdq{}}\PY{l+s+s2}{Age}\PY{l+s+s2}{\PYZdq{}}\PY{p}{]}\PY{o}{.}\PY{n}{isnull}\PY{p}{(}\PY{p}{)}
\PY{n}{missing\PYZus{}cabin} \PY{o}{=} \PY{n}{dataset}\PY{p}{[}\PY{l+s+s2}{\PYZdq{}}\PY{l+s+s2}{Cabin}\PY{l+s+s2}{\PYZdq{}}\PY{p}{]}\PY{o}{.}\PY{n}{isnull}\PY{p}{(}\PY{p}{)}

\PY{c+c1}{\PYZsh{} Find those passengers missing both}
\PY{n}{unknown\PYZus{}age\PYZus{}and\PYZus{}cabin} \PY{o}{=} \PY{n}{dataset}\PY{p}{[}\PY{n}{missing\PYZus{}age} \PY{o}{\PYZam{}} \PY{n}{missing\PYZus{}cabin}\PY{p}{]}
\PY{n+nb}{print}\PY{p}{(}\PY{l+s+s2}{\PYZdq{}}\PY{l+s+s2}{Number of passengers missing age and cabin information:}\PY{l+s+s2}{\PYZdq{}}\PY{p}{,} \PY{n+nb}{len}\PY{p}{(}\PY{n}{unknown\PYZus{}age\PYZus{}and\PYZus{}cabin}\PY{p}{)}\PY{p}{)}
\end{Verbatim}
\end{tcolorbox}

    \begin{Verbatim}[commandchars=\\\{\}]
Number of passengers missing age and cabin information: 158
    \end{Verbatim}

    Our suspicions were correct - most passengers missing age information
are also missing cabin information.

Normally, from here, we would want to know \emph{why} we have this
issue. A good hypothesis is that information was not collected carefully
enough for the passengers who used the cheap tickets.

Let's plot a histogram of ticket classes, and another of just people
missing information.

    \begin{tcolorbox}[breakable, size=fbox, boxrule=1pt, pad at break*=1mm,colback=cellbackground, colframe=cellborder]
\prompt{In}{incolor}{21}{\boxspacing}
\begin{Verbatim}[commandchars=\\\{\}]
\PY{k+kn}{import} \PY{n+nn}{graphing}

\PY{c+c1}{\PYZsh{} The \PYZsq{}graphing\PYZsq{} library is custom code we use to make graphs}
\PY{c+c1}{\PYZsh{} quickly. If you don\PYZsq{}t run this notebook in the sandbox}
\PY{c+c1}{\PYZsh{} environment, you might need to formally install this library}
\PY{c+c1}{\PYZsh{} in the environment you use. See the first cell of this notebook}
\PY{c+c1}{\PYZsh{} for more information about installation of the \PYZsq{}graphing\PYZsq{}}
\PY{c+c1}{\PYZsh{} library.}
\PY{c+c1}{\PYZsh{}}
\PY{c+c1}{\PYZsh{} To review the \PYZsq{}graphing\PYZsq{} library in detail, find it in our}
\PY{c+c1}{\PYZsh{} GitHub repository}

\PY{n}{graphing}\PY{o}{.}\PY{n}{histogram}\PY{p}{(}\PY{n}{dataset}\PY{p}{,} \PY{l+s+s1}{\PYZsq{}}\PY{l+s+s1}{Pclass}\PY{l+s+s1}{\PYZsq{}}\PY{p}{,} \PY{n}{title}\PY{o}{=}\PY{l+s+s1}{\PYZsq{}}\PY{l+s+s1}{Ticket Class (All Passengers)}\PY{l+s+s1}{\PYZsq{}}\PY{p}{,} \PY{n}{show}\PY{o}{=}\PY{k+kc}{True}\PY{p}{)}
\PY{n}{graphing}\PY{o}{.}\PY{n}{histogram}\PY{p}{(}\PY{n}{unknown\PYZus{}age\PYZus{}and\PYZus{}cabin}\PY{p}{,} \PY{l+s+s1}{\PYZsq{}}\PY{l+s+s1}{Pclass}\PY{l+s+s1}{\PYZsq{}}\PY{p}{,} \PY{n}{title}\PY{o}{=}\PY{l+s+s1}{\PYZsq{}}\PY{l+s+s1}{Ticket Class (Passengers Missing Cabin and Age Information)}\PY{l+s+s1}{\PYZsq{}}\PY{p}{)}
\end{Verbatim}
\end{tcolorbox}

    
    
    
    
    It seems that those passengers with missing information typically used
the cheaper tickets. These sorts of biases might cause problems in
real-world analyses.

\hypertarget{missing-as-zero}{%
\subsection{Missing as Zero}\label{missing-as-zero}}

Additionally, some datasets may have missing values that appear as zero.
While the Titanic dataset doesn't have this problem, let's see how that
would work here.

    \begin{tcolorbox}[breakable, size=fbox, boxrule=1pt, pad at break*=1mm,colback=cellbackground, colframe=cellborder]
\prompt{In}{incolor}{22}{\boxspacing}
\begin{Verbatim}[commandchars=\\\{\}]
\PY{k+kn}{import} \PY{n+nn}{numpy} \PY{k}{as} \PY{n+nn}{np}

\PY{c+c1}{\PYZsh{} Print out the average age of passengers for whom we have age data}
\PY{n}{mean\PYZus{}age} \PY{o}{=} \PY{n}{np}\PY{o}{.}\PY{n}{mean}\PY{p}{(}\PY{n}{dataset}\PY{o}{.}\PY{n}{Age}\PY{p}{)}
\PY{n+nb}{print}\PY{p}{(}\PY{l+s+s2}{\PYZdq{}}\PY{l+s+s2}{The average age on the ship was}\PY{l+s+s2}{\PYZdq{}}\PY{p}{,} \PY{n}{mean\PYZus{}age}\PY{p}{,} \PY{l+s+s2}{\PYZdq{}}\PY{l+s+s2}{years old}\PY{l+s+s2}{\PYZdq{}}\PY{p}{)}

\PY{c+c1}{\PYZsh{} Now, make another model where missing ages contained a \PYZsq{}0\PYZsq{}}
\PY{n}{dataset}\PY{p}{[}\PY{l+s+s1}{\PYZsq{}}\PY{l+s+s1}{Age\PYZus{}2}\PY{l+s+s1}{\PYZsq{}}\PY{p}{]} \PY{o}{=} \PY{n}{dataset}\PY{p}{[}\PY{l+s+s1}{\PYZsq{}}\PY{l+s+s1}{Age}\PY{l+s+s1}{\PYZsq{}}\PY{p}{]}\PY{o}{.}\PY{n}{fillna}\PY{p}{(}\PY{l+m+mi}{0}\PY{p}{)}
\PY{n}{mean\PYZus{}age} \PY{o}{=} \PY{n}{np}\PY{o}{.}\PY{n}{mean}\PY{p}{(}\PY{n}{dataset}\PY{o}{.}\PY{n}{Age\PYZus{}2}\PY{p}{)}
\PY{n+nb}{print}\PY{p}{(}\PY{l+s+s2}{\PYZdq{}}\PY{l+s+s2}{The average age on the ship was}\PY{l+s+s2}{\PYZdq{}}\PY{p}{,} \PY{n}{mean\PYZus{}age}\PY{p}{,} \PY{l+s+s2}{\PYZdq{}}\PY{l+s+s2}{years old}\PY{l+s+s2}{\PYZdq{}}\PY{p}{)}
\end{Verbatim}
\end{tcolorbox}

    \begin{Verbatim}[commandchars=\\\{\}]
The average age on the ship was 29.69911764705882 years old
The average age on the ship was 23.79929292929293 years old
    \end{Verbatim}

    What happened here? Our analyses have considered the values of
\texttt{0} to not be `missing' but rather to be actual ages.

This shows that it can be important to time the review of your raw data
before you run the analyses. Another fast way to get a feel for a
dataset is to graph its distribution:

    \begin{tcolorbox}[breakable, size=fbox, boxrule=1pt, pad at break*=1mm,colback=cellbackground, colframe=cellborder]
\prompt{In}{incolor}{23}{\boxspacing}
\begin{Verbatim}[commandchars=\\\{\}]
\PY{n}{graphing}\PY{o}{.}\PY{n}{histogram}\PY{p}{(}\PY{n}{dataset}\PY{p}{,} \PY{n}{label\PYZus{}x}\PY{o}{=}\PY{l+s+s2}{\PYZdq{}}\PY{l+s+s2}{Age\PYZus{}2}\PY{l+s+s2}{\PYZdq{}}\PY{p}{)}
\end{Verbatim}
\end{tcolorbox}

    
    
    Here, we see an unlikely number of very young children. This would be
cause for further inspection of the data, to hopefully spot the fact
that the missing ages appear as zeros.

\hypertarget{handling-missing-data}{%
\subsection{Handling Missing Data}\label{handling-missing-data}}

There are many ways to address missing data, each with pros and cons.

Let's take a look at the less complex options:

\hypertarget{option-1-delete-data-with-missing-rows}{%
\subsubsection{Option 1: Delete data with missing
rows}\label{option-1-delete-data-with-missing-rows}}

When we have a model that cannot handle missing data, the most prudent
thing to do is to remove rows that have information missing.

Let's remove some data from the \texttt{Embarked} column, which only has
two rows with missing data.

    \begin{tcolorbox}[breakable, size=fbox, boxrule=1pt, pad at break*=1mm,colback=cellbackground, colframe=cellborder]
\prompt{In}{incolor}{24}{\boxspacing}
\begin{Verbatim}[commandchars=\\\{\}]
\PY{c+c1}{\PYZsh{} Create a \PYZdq{}clean\PYZdq{} dataset, where we cumulatively fix missing values}
\PY{c+c1}{\PYZsh{} Start by removing rows ONLY where \PYZdq{}Embarked\PYZdq{} has no values}
\PY{n+nb}{print}\PY{p}{(}\PY{l+s+sa}{f}\PY{l+s+s2}{\PYZdq{}}\PY{l+s+s2}{The original size of our dataset was}\PY{l+s+s2}{\PYZdq{}}\PY{p}{,} \PY{n}{dataset}\PY{o}{.}\PY{n}{shape}\PY{p}{)}
\PY{n}{clean\PYZus{}dataset} \PY{o}{=} \PY{n}{dataset}\PY{o}{.}\PY{n}{dropna}\PY{p}{(}\PY{n}{subset}\PY{o}{=}\PY{p}{[}\PY{l+s+s2}{\PYZdq{}}\PY{l+s+s2}{Embarked}\PY{l+s+s2}{\PYZdq{}}\PY{p}{]}\PY{p}{)}
\PY{n}{clean\PYZus{}dataset} \PY{o}{=} \PY{n}{clean\PYZus{}dataset}\PY{o}{.}\PY{n}{reindex}\PY{p}{(}\PY{p}{)}

\PY{c+c1}{\PYZsh{} How many rows do we have now?}
\PY{n+nb}{print}\PY{p}{(}\PY{l+s+s2}{\PYZdq{}}\PY{l+s+s2}{The shape for the clean dataset is}\PY{l+s+s2}{\PYZdq{}}\PY{p}{,} \PY{n}{clean\PYZus{}dataset}\PY{o}{.}\PY{n}{shape}\PY{p}{)}
\end{Verbatim}
\end{tcolorbox}

    \begin{Verbatim}[commandchars=\\\{\}]
The original size of our dataset was (891, 13)
The shape for the clean dataset is (889, 13)
    \end{Verbatim}

    We can see that this removed the offending two rows from our new, clean
dataset.

\hypertarget{option-2-replace-empty-values-with-the-mean-or-median-for-that-data.}{%
\subsubsection{Option 2: Replace empty values with the mean or median
for that
data.}\label{option-2-replace-empty-values-with-the-mean-or-median-for-that-data.}}

Sometimes, our model cannot handle missing values, and we also cannot
afford to remove too much data. In this case, we can sometimes fill in
missing data with an average calculated on the basis of the rest of the
dataset. Note that imputing data like this can affect model performance
in a negative way. Usually, it's better to simply remove missing data,
or to use a model designed to handle missing values.

Below, we impute data for the \texttt{Age} field. We use the mean
\texttt{Age} from the remaining rows, given that \textgreater{}80\% of
these aren't empty:

    \begin{tcolorbox}[breakable, size=fbox, boxrule=1pt, pad at break*=1mm,colback=cellbackground, colframe=cellborder]
\prompt{In}{incolor}{25}{\boxspacing}
\begin{Verbatim}[commandchars=\\\{\}]
\PY{c+c1}{\PYZsh{} Calculate the mean value for the Age column}
\PY{n}{mean\PYZus{}age} \PY{o}{=} \PY{n}{clean\PYZus{}dataset}\PY{p}{[}\PY{l+s+s2}{\PYZdq{}}\PY{l+s+s2}{Age}\PY{l+s+s2}{\PYZdq{}}\PY{p}{]}\PY{o}{.}\PY{n}{mean}\PY{p}{(}\PY{p}{)}

\PY{n+nb}{print}\PY{p}{(}\PY{l+s+s2}{\PYZdq{}}\PY{l+s+s2}{The mean age is}\PY{l+s+s2}{\PYZdq{}}\PY{p}{,} \PY{n}{mean\PYZus{}age}\PY{p}{)}

\PY{c+c1}{\PYZsh{} Replace empty values in \PYZdq{}Age\PYZdq{} with the mean calculated above}
\PY{n}{clean\PYZus{}dataset}\PY{p}{[}\PY{l+s+s2}{\PYZdq{}}\PY{l+s+s2}{Age}\PY{l+s+s2}{\PYZdq{}}\PY{p}{]}\PY{o}{.}\PY{n}{fillna}\PY{p}{(}\PY{n}{mean\PYZus{}age}\PY{p}{,} \PY{n}{inplace}\PY{o}{=}\PY{k+kc}{True}\PY{p}{)}

\PY{c+c1}{\PYZsh{} Let\PYZsq{}s see what the clean dataset looks like now}
\PY{n+nb}{print}\PY{p}{(}\PY{n}{clean\PYZus{}dataset}\PY{o}{.}\PY{n}{isnull}\PY{p}{(}\PY{p}{)}\PY{o}{.}\PY{n}{sum}\PY{p}{(}\PY{p}{)}\PY{o}{.}\PY{n}{to\PYZus{}frame}\PY{p}{(}\PY{p}{)}\PY{o}{.}\PY{n}{rename}\PY{p}{(}\PY{n}{columns}\PY{o}{=}\PY{p}{\PYZob{}}\PY{l+m+mi}{0}\PY{p}{:}\PY{l+s+s1}{\PYZsq{}}\PY{l+s+s1}{Empty Cells}\PY{l+s+s1}{\PYZsq{}}\PY{p}{\PYZcb{}}\PY{p}{)}\PY{p}{)}
\end{Verbatim}
\end{tcolorbox}

    \begin{Verbatim}[commandchars=\\\{\}]
The mean age is 29.64209269662921
             Empty Cells
PassengerId            0
Survived               0
Pclass                 0
Name                   0
Sex                    0
Age                    0
SibSp                  0
Parch                  0
Ticket                 0
Fare                   0
Cabin                687
Embarked               0
Age\_2                  0
    \end{Verbatim}

    The \texttt{Age} field has no longer has empty cells anymore.

\hypertarget{option-3-assign-a-new-category-to-unknown-categorical-data}{%
\subsubsection{Option 3: Assign a new category to unknown categorical
data}\label{option-3-assign-a-new-category-to-unknown-categorical-data}}

The \texttt{Cabin} field is a categorical field, because the Titanic
cabins have a finite number of possible options. Unfortunately, many
records have no cabin listed.

For this exercise, it makes perfect sense to create an \texttt{Unknown}
category, and assign it to the cases where the cabin is unknown:

    \begin{tcolorbox}[breakable, size=fbox, boxrule=1pt, pad at break*=1mm,colback=cellbackground, colframe=cellborder]
\prompt{In}{incolor}{26}{\boxspacing}
\begin{Verbatim}[commandchars=\\\{\}]
\PY{c+c1}{\PYZsh{} Assign unknown to records where \PYZdq{}Cabin\PYZdq{} is empty}
\PY{n}{clean\PYZus{}dataset}\PY{p}{[}\PY{l+s+s2}{\PYZdq{}}\PY{l+s+s2}{Cabin}\PY{l+s+s2}{\PYZdq{}}\PY{p}{]}\PY{o}{.}\PY{n}{fillna}\PY{p}{(}\PY{l+s+s2}{\PYZdq{}}\PY{l+s+s2}{Unknown}\PY{l+s+s2}{\PYZdq{}}\PY{p}{,} \PY{n}{inplace}\PY{o}{=}\PY{k+kc}{True}\PY{p}{)}

\PY{c+c1}{\PYZsh{} Let\PYZsq{}s see what the clean dataset looks like now}
\PY{n+nb}{print}\PY{p}{(}\PY{n}{clean\PYZus{}dataset}\PY{o}{.}\PY{n}{isnull}\PY{p}{(}\PY{p}{)}\PY{o}{.}\PY{n}{sum}\PY{p}{(}\PY{p}{)}\PY{o}{.}\PY{n}{to\PYZus{}frame}\PY{p}{(}\PY{p}{)}\PY{o}{.}\PY{n}{rename}\PY{p}{(}\PY{n}{columns}\PY{o}{=}\PY{p}{\PYZob{}}\PY{l+m+mi}{0}\PY{p}{:}\PY{l+s+s1}{\PYZsq{}}\PY{l+s+s1}{Empty Cells}\PY{l+s+s1}{\PYZsq{}}\PY{p}{\PYZcb{}}\PY{p}{)}\PY{p}{)}
\end{Verbatim}
\end{tcolorbox}

    \begin{Verbatim}[commandchars=\\\{\}]
             Empty Cells
PassengerId            0
Survived               0
Pclass                 0
Name                   0
Sex                    0
Age                    0
SibSp                  0
Parch                  0
Ticket                 0
Fare                   0
Cabin                  0
Embarked               0
Age\_2                  0
    \end{Verbatim}

    That's it! No more missing data!

We only lost two records (where \texttt{Embarked} was empty).

That said, we had to make some approximations to fill the missing gaps
for the \texttt{Age} and \texttt{Cabin} columns, and those will
certainly influence the performance of any model we train on this data.

\hypertarget{summary}{%
\subsection{Summary}\label{summary}}

Missing values can affect the way a Machine Learning model works in a
negative way. It's important to quickly verify the existence of data
gaps, and the locations of those gaps.

You can now get a ``big picture'' of what is missing, and select only
those items that you must address, by the use of lists and charts.

In this exercise, we practiced:

\begin{itemize}
\tightlist
\item
  Finding and visualization of missing dataset values, using the
  \texttt{pandas} and \texttt{missingno} packages.
\item
  Checking whether a dataset uses the value `0' to represent missing
  values.
\item
  Handling missing data in three ways: removing of rows that contain
  missing values, replacment of the missing values with the mean or
  median of that particular feature, and creation of a new
  \texttt{Unknown} category, if dealing with categorical data.
\end{itemize}


    % Add a bibliography block to the postdoc
    
    
    
\end{document}
