\documentclass[11pt]{article}

    \usepackage[breakable]{tcolorbox}
    \usepackage{parskip} % Stop auto-indenting (to mimic markdown behaviour)
    

    % Basic figure setup, for now with no caption control since it's done
    % automatically by Pandoc (which extracts ![](path) syntax from Markdown).
    \usepackage{graphicx}
    % Maintain compatibility with old templates. Remove in nbconvert 6.0
    \let\Oldincludegraphics\includegraphics
    % Ensure that by default, figures have no caption (until we provide a
    % proper Figure object with a Caption API and a way to capture that
    % in the conversion process - todo).
    \usepackage{caption}
    \DeclareCaptionFormat{nocaption}{}
    \captionsetup{format=nocaption,aboveskip=0pt,belowskip=0pt}

    \usepackage{float}
    \floatplacement{figure}{H} % forces figures to be placed at the correct location
    \usepackage{xcolor} % Allow colors to be defined
    \usepackage{enumerate} % Needed for markdown enumerations to work
    \usepackage{geometry} % Used to adjust the document margins
    \usepackage{amsmath} % Equations
    \usepackage{amssymb} % Equations
    \usepackage{textcomp} % defines textquotesingle
    % Hack from http://tex.stackexchange.com/a/47451/13684:
    \AtBeginDocument{%
        \def\PYZsq{\textquotesingle}% Upright quotes in Pygmentized code
    }
    \usepackage{upquote} % Upright quotes for verbatim code
    \usepackage{eurosym} % defines \euro

    \usepackage{iftex}
    \ifPDFTeX
        \usepackage[T1]{fontenc}
        \IfFileExists{alphabeta.sty}{
              \usepackage{alphabeta}
          }{
              \usepackage[mathletters]{ucs}
              \usepackage[utf8x]{inputenc}
          }
    \else
        \usepackage{fontspec}
        \usepackage{unicode-math}
    \fi

    \usepackage{fancyvrb} % verbatim replacement that allows latex
    \usepackage{grffile} % extends the file name processing of package graphics
                         % to support a larger range
    \makeatletter % fix for old versions of grffile with XeLaTeX
    \@ifpackagelater{grffile}{2019/11/01}
    {
      % Do nothing on new versions
    }
    {
      \def\Gread@@xetex#1{%
        \IfFileExists{"\Gin@base".bb}%
        {\Gread@eps{\Gin@base.bb}}%
        {\Gread@@xetex@aux#1}%
      }
    }
    \makeatother
    \usepackage[Export]{adjustbox} % Used to constrain images to a maximum size
    \adjustboxset{max size={0.9\linewidth}{0.9\paperheight}}

    % The hyperref package gives us a pdf with properly built
    % internal navigation ('pdf bookmarks' for the table of contents,
    % internal cross-reference links, web links for URLs, etc.)
    \usepackage{hyperref}
    % The default LaTeX title has an obnoxious amount of whitespace. By default,
    % titling removes some of it. It also provides customization options.
    \usepackage{titling}
    \usepackage{longtable} % longtable support required by pandoc >1.10
    \usepackage{booktabs}  % table support for pandoc > 1.12.2
    \usepackage{array}     % table support for pandoc >= 2.11.3
    \usepackage{calc}      % table minipage width calculation for pandoc >= 2.11.1
    \usepackage[inline]{enumitem} % IRkernel/repr support (it uses the enumerate* environment)
    \usepackage[normalem]{ulem} % ulem is needed to support strikethroughs (\sout)
                                % normalem makes italics be italics, not underlines
    \usepackage{mathrsfs}
    

    
    % Colors for the hyperref package
    \definecolor{urlcolor}{rgb}{0,.145,.698}
    \definecolor{linkcolor}{rgb}{.71,0.21,0.01}
    \definecolor{citecolor}{rgb}{.12,.54,.11}

    % ANSI colors
    \definecolor{ansi-black}{HTML}{3E424D}
    \definecolor{ansi-black-intense}{HTML}{282C36}
    \definecolor{ansi-red}{HTML}{E75C58}
    \definecolor{ansi-red-intense}{HTML}{B22B31}
    \definecolor{ansi-green}{HTML}{00A250}
    \definecolor{ansi-green-intense}{HTML}{007427}
    \definecolor{ansi-yellow}{HTML}{DDB62B}
    \definecolor{ansi-yellow-intense}{HTML}{B27D12}
    \definecolor{ansi-blue}{HTML}{208FFB}
    \definecolor{ansi-blue-intense}{HTML}{0065CA}
    \definecolor{ansi-magenta}{HTML}{D160C4}
    \definecolor{ansi-magenta-intense}{HTML}{A03196}
    \definecolor{ansi-cyan}{HTML}{60C6C8}
    \definecolor{ansi-cyan-intense}{HTML}{258F8F}
    \definecolor{ansi-white}{HTML}{C5C1B4}
    \definecolor{ansi-white-intense}{HTML}{A1A6B2}
    \definecolor{ansi-default-inverse-fg}{HTML}{FFFFFF}
    \definecolor{ansi-default-inverse-bg}{HTML}{000000}

    % common color for the border for error outputs.
    \definecolor{outerrorbackground}{HTML}{FFDFDF}

    % commands and environments needed by pandoc snippets
    % extracted from the output of `pandoc -s`
    \providecommand{\tightlist}{%
      \setlength{\itemsep}{0pt}\setlength{\parskip}{0pt}}
    \DefineVerbatimEnvironment{Highlighting}{Verbatim}{commandchars=\\\{\}}
    % Add ',fontsize=\small' for more characters per line
    \newenvironment{Shaded}{}{}
    \newcommand{\KeywordTok}[1]{\textcolor[rgb]{0.00,0.44,0.13}{\textbf{{#1}}}}
    \newcommand{\DataTypeTok}[1]{\textcolor[rgb]{0.56,0.13,0.00}{{#1}}}
    \newcommand{\DecValTok}[1]{\textcolor[rgb]{0.25,0.63,0.44}{{#1}}}
    \newcommand{\BaseNTok}[1]{\textcolor[rgb]{0.25,0.63,0.44}{{#1}}}
    \newcommand{\FloatTok}[1]{\textcolor[rgb]{0.25,0.63,0.44}{{#1}}}
    \newcommand{\CharTok}[1]{\textcolor[rgb]{0.25,0.44,0.63}{{#1}}}
    \newcommand{\StringTok}[1]{\textcolor[rgb]{0.25,0.44,0.63}{{#1}}}
    \newcommand{\CommentTok}[1]{\textcolor[rgb]{0.38,0.63,0.69}{\textit{{#1}}}}
    \newcommand{\OtherTok}[1]{\textcolor[rgb]{0.00,0.44,0.13}{{#1}}}
    \newcommand{\AlertTok}[1]{\textcolor[rgb]{1.00,0.00,0.00}{\textbf{{#1}}}}
    \newcommand{\FunctionTok}[1]{\textcolor[rgb]{0.02,0.16,0.49}{{#1}}}
    \newcommand{\RegionMarkerTok}[1]{{#1}}
    \newcommand{\ErrorTok}[1]{\textcolor[rgb]{1.00,0.00,0.00}{\textbf{{#1}}}}
    \newcommand{\NormalTok}[1]{{#1}}

    % Additional commands for more recent versions of Pandoc
    \newcommand{\ConstantTok}[1]{\textcolor[rgb]{0.53,0.00,0.00}{{#1}}}
    \newcommand{\SpecialCharTok}[1]{\textcolor[rgb]{0.25,0.44,0.63}{{#1}}}
    \newcommand{\VerbatimStringTok}[1]{\textcolor[rgb]{0.25,0.44,0.63}{{#1}}}
    \newcommand{\SpecialStringTok}[1]{\textcolor[rgb]{0.73,0.40,0.53}{{#1}}}
    \newcommand{\ImportTok}[1]{{#1}}
    \newcommand{\DocumentationTok}[1]{\textcolor[rgb]{0.73,0.13,0.13}{\textit{{#1}}}}
    \newcommand{\AnnotationTok}[1]{\textcolor[rgb]{0.38,0.63,0.69}{\textbf{\textit{{#1}}}}}
    \newcommand{\CommentVarTok}[1]{\textcolor[rgb]{0.38,0.63,0.69}{\textbf{\textit{{#1}}}}}
    \newcommand{\VariableTok}[1]{\textcolor[rgb]{0.10,0.09,0.49}{{#1}}}
    \newcommand{\ControlFlowTok}[1]{\textcolor[rgb]{0.00,0.44,0.13}{\textbf{{#1}}}}
    \newcommand{\OperatorTok}[1]{\textcolor[rgb]{0.40,0.40,0.40}{{#1}}}
    \newcommand{\BuiltInTok}[1]{{#1}}
    \newcommand{\ExtensionTok}[1]{{#1}}
    \newcommand{\PreprocessorTok}[1]{\textcolor[rgb]{0.74,0.48,0.00}{{#1}}}
    \newcommand{\AttributeTok}[1]{\textcolor[rgb]{0.49,0.56,0.16}{{#1}}}
    \newcommand{\InformationTok}[1]{\textcolor[rgb]{0.38,0.63,0.69}{\textbf{\textit{{#1}}}}}
    \newcommand{\WarningTok}[1]{\textcolor[rgb]{0.38,0.63,0.69}{\textbf{\textit{{#1}}}}}


    % Define a nice break command that doesn't care if a line doesn't already
    % exist.
    \def\br{\hspace*{\fill} \\* }
    % Math Jax compatibility definitions
    \def\gt{>}
    \def\lt{<}
    \let\Oldtex\TeX
    \let\Oldlatex\LaTeX
    \renewcommand{\TeX}{\textrm{\Oldtex}}
    \renewcommand{\LaTeX}{\textrm{\Oldlatex}}
    % Document parameters
    % Document title
    \title{Notebook}
    
    
    
    
    
    
    
% Pygments definitions
\makeatletter
\def\PY@reset{\let\PY@it=\relax \let\PY@bf=\relax%
    \let\PY@ul=\relax \let\PY@tc=\relax%
    \let\PY@bc=\relax \let\PY@ff=\relax}
\def\PY@tok#1{\csname PY@tok@#1\endcsname}
\def\PY@toks#1+{\ifx\relax#1\empty\else%
    \PY@tok{#1}\expandafter\PY@toks\fi}
\def\PY@do#1{\PY@bc{\PY@tc{\PY@ul{%
    \PY@it{\PY@bf{\PY@ff{#1}}}}}}}
\def\PY#1#2{\PY@reset\PY@toks#1+\relax+\PY@do{#2}}

\@namedef{PY@tok@w}{\def\PY@tc##1{\textcolor[rgb]{0.73,0.73,0.73}{##1}}}
\@namedef{PY@tok@c}{\let\PY@it=\textit\def\PY@tc##1{\textcolor[rgb]{0.24,0.48,0.48}{##1}}}
\@namedef{PY@tok@cp}{\def\PY@tc##1{\textcolor[rgb]{0.61,0.40,0.00}{##1}}}
\@namedef{PY@tok@k}{\let\PY@bf=\textbf\def\PY@tc##1{\textcolor[rgb]{0.00,0.50,0.00}{##1}}}
\@namedef{PY@tok@kp}{\def\PY@tc##1{\textcolor[rgb]{0.00,0.50,0.00}{##1}}}
\@namedef{PY@tok@kt}{\def\PY@tc##1{\textcolor[rgb]{0.69,0.00,0.25}{##1}}}
\@namedef{PY@tok@o}{\def\PY@tc##1{\textcolor[rgb]{0.40,0.40,0.40}{##1}}}
\@namedef{PY@tok@ow}{\let\PY@bf=\textbf\def\PY@tc##1{\textcolor[rgb]{0.67,0.13,1.00}{##1}}}
\@namedef{PY@tok@nb}{\def\PY@tc##1{\textcolor[rgb]{0.00,0.50,0.00}{##1}}}
\@namedef{PY@tok@nf}{\def\PY@tc##1{\textcolor[rgb]{0.00,0.00,1.00}{##1}}}
\@namedef{PY@tok@nc}{\let\PY@bf=\textbf\def\PY@tc##1{\textcolor[rgb]{0.00,0.00,1.00}{##1}}}
\@namedef{PY@tok@nn}{\let\PY@bf=\textbf\def\PY@tc##1{\textcolor[rgb]{0.00,0.00,1.00}{##1}}}
\@namedef{PY@tok@ne}{\let\PY@bf=\textbf\def\PY@tc##1{\textcolor[rgb]{0.80,0.25,0.22}{##1}}}
\@namedef{PY@tok@nv}{\def\PY@tc##1{\textcolor[rgb]{0.10,0.09,0.49}{##1}}}
\@namedef{PY@tok@no}{\def\PY@tc##1{\textcolor[rgb]{0.53,0.00,0.00}{##1}}}
\@namedef{PY@tok@nl}{\def\PY@tc##1{\textcolor[rgb]{0.46,0.46,0.00}{##1}}}
\@namedef{PY@tok@ni}{\let\PY@bf=\textbf\def\PY@tc##1{\textcolor[rgb]{0.44,0.44,0.44}{##1}}}
\@namedef{PY@tok@na}{\def\PY@tc##1{\textcolor[rgb]{0.41,0.47,0.13}{##1}}}
\@namedef{PY@tok@nt}{\let\PY@bf=\textbf\def\PY@tc##1{\textcolor[rgb]{0.00,0.50,0.00}{##1}}}
\@namedef{PY@tok@nd}{\def\PY@tc##1{\textcolor[rgb]{0.67,0.13,1.00}{##1}}}
\@namedef{PY@tok@s}{\def\PY@tc##1{\textcolor[rgb]{0.73,0.13,0.13}{##1}}}
\@namedef{PY@tok@sd}{\let\PY@it=\textit\def\PY@tc##1{\textcolor[rgb]{0.73,0.13,0.13}{##1}}}
\@namedef{PY@tok@si}{\let\PY@bf=\textbf\def\PY@tc##1{\textcolor[rgb]{0.64,0.35,0.47}{##1}}}
\@namedef{PY@tok@se}{\let\PY@bf=\textbf\def\PY@tc##1{\textcolor[rgb]{0.67,0.36,0.12}{##1}}}
\@namedef{PY@tok@sr}{\def\PY@tc##1{\textcolor[rgb]{0.64,0.35,0.47}{##1}}}
\@namedef{PY@tok@ss}{\def\PY@tc##1{\textcolor[rgb]{0.10,0.09,0.49}{##1}}}
\@namedef{PY@tok@sx}{\def\PY@tc##1{\textcolor[rgb]{0.00,0.50,0.00}{##1}}}
\@namedef{PY@tok@m}{\def\PY@tc##1{\textcolor[rgb]{0.40,0.40,0.40}{##1}}}
\@namedef{PY@tok@gh}{\let\PY@bf=\textbf\def\PY@tc##1{\textcolor[rgb]{0.00,0.00,0.50}{##1}}}
\@namedef{PY@tok@gu}{\let\PY@bf=\textbf\def\PY@tc##1{\textcolor[rgb]{0.50,0.00,0.50}{##1}}}
\@namedef{PY@tok@gd}{\def\PY@tc##1{\textcolor[rgb]{0.63,0.00,0.00}{##1}}}
\@namedef{PY@tok@gi}{\def\PY@tc##1{\textcolor[rgb]{0.00,0.52,0.00}{##1}}}
\@namedef{PY@tok@gr}{\def\PY@tc##1{\textcolor[rgb]{0.89,0.00,0.00}{##1}}}
\@namedef{PY@tok@ge}{\let\PY@it=\textit}
\@namedef{PY@tok@gs}{\let\PY@bf=\textbf}
\@namedef{PY@tok@gp}{\let\PY@bf=\textbf\def\PY@tc##1{\textcolor[rgb]{0.00,0.00,0.50}{##1}}}
\@namedef{PY@tok@go}{\def\PY@tc##1{\textcolor[rgb]{0.44,0.44,0.44}{##1}}}
\@namedef{PY@tok@gt}{\def\PY@tc##1{\textcolor[rgb]{0.00,0.27,0.87}{##1}}}
\@namedef{PY@tok@err}{\def\PY@bc##1{{\setlength{\fboxsep}{\string -\fboxrule}\fcolorbox[rgb]{1.00,0.00,0.00}{1,1,1}{\strut ##1}}}}
\@namedef{PY@tok@kc}{\let\PY@bf=\textbf\def\PY@tc##1{\textcolor[rgb]{0.00,0.50,0.00}{##1}}}
\@namedef{PY@tok@kd}{\let\PY@bf=\textbf\def\PY@tc##1{\textcolor[rgb]{0.00,0.50,0.00}{##1}}}
\@namedef{PY@tok@kn}{\let\PY@bf=\textbf\def\PY@tc##1{\textcolor[rgb]{0.00,0.50,0.00}{##1}}}
\@namedef{PY@tok@kr}{\let\PY@bf=\textbf\def\PY@tc##1{\textcolor[rgb]{0.00,0.50,0.00}{##1}}}
\@namedef{PY@tok@bp}{\def\PY@tc##1{\textcolor[rgb]{0.00,0.50,0.00}{##1}}}
\@namedef{PY@tok@fm}{\def\PY@tc##1{\textcolor[rgb]{0.00,0.00,1.00}{##1}}}
\@namedef{PY@tok@vc}{\def\PY@tc##1{\textcolor[rgb]{0.10,0.09,0.49}{##1}}}
\@namedef{PY@tok@vg}{\def\PY@tc##1{\textcolor[rgb]{0.10,0.09,0.49}{##1}}}
\@namedef{PY@tok@vi}{\def\PY@tc##1{\textcolor[rgb]{0.10,0.09,0.49}{##1}}}
\@namedef{PY@tok@vm}{\def\PY@tc##1{\textcolor[rgb]{0.10,0.09,0.49}{##1}}}
\@namedef{PY@tok@sa}{\def\PY@tc##1{\textcolor[rgb]{0.73,0.13,0.13}{##1}}}
\@namedef{PY@tok@sb}{\def\PY@tc##1{\textcolor[rgb]{0.73,0.13,0.13}{##1}}}
\@namedef{PY@tok@sc}{\def\PY@tc##1{\textcolor[rgb]{0.73,0.13,0.13}{##1}}}
\@namedef{PY@tok@dl}{\def\PY@tc##1{\textcolor[rgb]{0.73,0.13,0.13}{##1}}}
\@namedef{PY@tok@s2}{\def\PY@tc##1{\textcolor[rgb]{0.73,0.13,0.13}{##1}}}
\@namedef{PY@tok@sh}{\def\PY@tc##1{\textcolor[rgb]{0.73,0.13,0.13}{##1}}}
\@namedef{PY@tok@s1}{\def\PY@tc##1{\textcolor[rgb]{0.73,0.13,0.13}{##1}}}
\@namedef{PY@tok@mb}{\def\PY@tc##1{\textcolor[rgb]{0.40,0.40,0.40}{##1}}}
\@namedef{PY@tok@mf}{\def\PY@tc##1{\textcolor[rgb]{0.40,0.40,0.40}{##1}}}
\@namedef{PY@tok@mh}{\def\PY@tc##1{\textcolor[rgb]{0.40,0.40,0.40}{##1}}}
\@namedef{PY@tok@mi}{\def\PY@tc##1{\textcolor[rgb]{0.40,0.40,0.40}{##1}}}
\@namedef{PY@tok@il}{\def\PY@tc##1{\textcolor[rgb]{0.40,0.40,0.40}{##1}}}
\@namedef{PY@tok@mo}{\def\PY@tc##1{\textcolor[rgb]{0.40,0.40,0.40}{##1}}}
\@namedef{PY@tok@ch}{\let\PY@it=\textit\def\PY@tc##1{\textcolor[rgb]{0.24,0.48,0.48}{##1}}}
\@namedef{PY@tok@cm}{\let\PY@it=\textit\def\PY@tc##1{\textcolor[rgb]{0.24,0.48,0.48}{##1}}}
\@namedef{PY@tok@cpf}{\let\PY@it=\textit\def\PY@tc##1{\textcolor[rgb]{0.24,0.48,0.48}{##1}}}
\@namedef{PY@tok@c1}{\let\PY@it=\textit\def\PY@tc##1{\textcolor[rgb]{0.24,0.48,0.48}{##1}}}
\@namedef{PY@tok@cs}{\let\PY@it=\textit\def\PY@tc##1{\textcolor[rgb]{0.24,0.48,0.48}{##1}}}

\def\PYZbs{\char`\\}
\def\PYZus{\char`\_}
\def\PYZob{\char`\{}
\def\PYZcb{\char`\}}
\def\PYZca{\char`\^}
\def\PYZam{\char`\&}
\def\PYZlt{\char`\<}
\def\PYZgt{\char`\>}
\def\PYZsh{\char`\#}
\def\PYZpc{\char`\%}
\def\PYZdl{\char`\$}
\def\PYZhy{\char`\-}
\def\PYZsq{\char`\'}
\def\PYZdq{\char`\"}
\def\PYZti{\char`\~}
% for compatibility with earlier versions
\def\PYZat{@}
\def\PYZlb{[}
\def\PYZrb{]}
\makeatother


    % For linebreaks inside Verbatim environment from package fancyvrb.
    \makeatletter
        \newbox\Wrappedcontinuationbox
        \newbox\Wrappedvisiblespacebox
        \newcommand*\Wrappedvisiblespace {\textcolor{red}{\textvisiblespace}}
        \newcommand*\Wrappedcontinuationsymbol {\textcolor{red}{\llap{\tiny$\m@th\hookrightarrow$}}}
        \newcommand*\Wrappedcontinuationindent {3ex }
        \newcommand*\Wrappedafterbreak {\kern\Wrappedcontinuationindent\copy\Wrappedcontinuationbox}
        % Take advantage of the already applied Pygments mark-up to insert
        % potential linebreaks for TeX processing.
        %        {, <, #, %, $, ' and ": go to next line.
        %        _, }, ^, &, >, - and ~: stay at end of broken line.
        % Use of \textquotesingle for straight quote.
        \newcommand*\Wrappedbreaksatspecials {%
            \def\PYGZus{\discretionary{\char`\_}{\Wrappedafterbreak}{\char`\_}}%
            \def\PYGZob{\discretionary{}{\Wrappedafterbreak\char`\{}{\char`\{}}%
            \def\PYGZcb{\discretionary{\char`\}}{\Wrappedafterbreak}{\char`\}}}%
            \def\PYGZca{\discretionary{\char`\^}{\Wrappedafterbreak}{\char`\^}}%
            \def\PYGZam{\discretionary{\char`\&}{\Wrappedafterbreak}{\char`\&}}%
            \def\PYGZlt{\discretionary{}{\Wrappedafterbreak\char`\<}{\char`\<}}%
            \def\PYGZgt{\discretionary{\char`\>}{\Wrappedafterbreak}{\char`\>}}%
            \def\PYGZsh{\discretionary{}{\Wrappedafterbreak\char`\#}{\char`\#}}%
            \def\PYGZpc{\discretionary{}{\Wrappedafterbreak\char`\%}{\char`\%}}%
            \def\PYGZdl{\discretionary{}{\Wrappedafterbreak\char`\$}{\char`\$}}%
            \def\PYGZhy{\discretionary{\char`\-}{\Wrappedafterbreak}{\char`\-}}%
            \def\PYGZsq{\discretionary{}{\Wrappedafterbreak\textquotesingle}{\textquotesingle}}%
            \def\PYGZdq{\discretionary{}{\Wrappedafterbreak\char`\"}{\char`\"}}%
            \def\PYGZti{\discretionary{\char`\~}{\Wrappedafterbreak}{\char`\~}}%
        }
        % Some characters . , ; ? ! / are not pygmentized.
        % This macro makes them "active" and they will insert potential linebreaks
        \newcommand*\Wrappedbreaksatpunct {%
            \lccode`\~`\.\lowercase{\def~}{\discretionary{\hbox{\char`\.}}{\Wrappedafterbreak}{\hbox{\char`\.}}}%
            \lccode`\~`\,\lowercase{\def~}{\discretionary{\hbox{\char`\,}}{\Wrappedafterbreak}{\hbox{\char`\,}}}%
            \lccode`\~`\;\lowercase{\def~}{\discretionary{\hbox{\char`\;}}{\Wrappedafterbreak}{\hbox{\char`\;}}}%
            \lccode`\~`\:\lowercase{\def~}{\discretionary{\hbox{\char`\:}}{\Wrappedafterbreak}{\hbox{\char`\:}}}%
            \lccode`\~`\?\lowercase{\def~}{\discretionary{\hbox{\char`\?}}{\Wrappedafterbreak}{\hbox{\char`\?}}}%
            \lccode`\~`\!\lowercase{\def~}{\discretionary{\hbox{\char`\!}}{\Wrappedafterbreak}{\hbox{\char`\!}}}%
            \lccode`\~`\/\lowercase{\def~}{\discretionary{\hbox{\char`\/}}{\Wrappedafterbreak}{\hbox{\char`\/}}}%
            \catcode`\.\active
            \catcode`\,\active
            \catcode`\;\active
            \catcode`\:\active
            \catcode`\?\active
            \catcode`\!\active
            \catcode`\/\active
            \lccode`\~`\~
        }
    \makeatother

    \let\OriginalVerbatim=\Verbatim
    \makeatletter
    \renewcommand{\Verbatim}[1][1]{%
        %\parskip\z@skip
        \sbox\Wrappedcontinuationbox {\Wrappedcontinuationsymbol}%
        \sbox\Wrappedvisiblespacebox {\FV@SetupFont\Wrappedvisiblespace}%
        \def\FancyVerbFormatLine ##1{\hsize\linewidth
            \vtop{\raggedright\hyphenpenalty\z@\exhyphenpenalty\z@
                \doublehyphendemerits\z@\finalhyphendemerits\z@
                \strut ##1\strut}%
        }%
        % If the linebreak is at a space, the latter will be displayed as visible
        % space at end of first line, and a continuation symbol starts next line.
        % Stretch/shrink are however usually zero for typewriter font.
        \def\FV@Space {%
            \nobreak\hskip\z@ plus\fontdimen3\font minus\fontdimen4\font
            \discretionary{\copy\Wrappedvisiblespacebox}{\Wrappedafterbreak}
            {\kern\fontdimen2\font}%
        }%

        % Allow breaks at special characters using \PYG... macros.
        \Wrappedbreaksatspecials
        % Breaks at punctuation characters . , ; ? ! and / need catcode=\active
        \OriginalVerbatim[#1,codes*=\Wrappedbreaksatpunct]%
    }
    \makeatother

    % Exact colors from NB
    \definecolor{incolor}{HTML}{303F9F}
    \definecolor{outcolor}{HTML}{D84315}
    \definecolor{cellborder}{HTML}{CFCFCF}
    \definecolor{cellbackground}{HTML}{F7F7F7}

    % prompt
    \makeatletter
    \newcommand{\boxspacing}{\kern\kvtcb@left@rule\kern\kvtcb@boxsep}
    \makeatother
    \newcommand{\prompt}[4]{
        {\ttfamily\llap{{\color{#2}[#3]:\hspace{3pt}#4}}\vspace{-\baselineskip}}
    }
    

    
    % Prevent overflowing lines due to hard-to-break entities
    \sloppy
    % Setup hyperref package
    \hypersetup{
      breaklinks=true,  % so long urls are correctly broken across lines
      colorlinks=true,
      urlcolor=urlcolor,
      linkcolor=linkcolor,
      citecolor=citecolor,
      }
    % Slightly bigger margins than the latex defaults
    
    \geometry{verbose,tmargin=1in,bmargin=1in,lmargin=1in,rmargin=1in}
    
    

\begin{document}
    
    \maketitle
    
    

    
    \hypertarget{exercise-titanic-dataset---visualising-different-types-of-data}{%
\section{Exercise: Titanic Dataset - Visualising Different Types of
Data}\label{exercise-titanic-dataset---visualising-different-types-of-data}}

To build better machine learning models we should understand the
available data. This usually means both:

\begin{enumerate}
\def\labelenumi{\arabic{enumi}.}
\item
  data visualization
\item
  understanding the kind of data we have available
\end{enumerate}

In this module, we'll practice cleaning our Titanic dataset, and
visualization of different kinds of data, especially

\begin{itemize}
\tightlist
\item
  continuous
\item
  ordinal
\item
  categorical
\item
  simple identity column
\end{itemize}

data types.

\hypertarget{a-note-on-our-graphing}{%
\subsection{A Note On Our Graphing}\label{a-note-on-our-graphing}}

In this exercise, we'll use a custom python script to create our graphs.
This script uses Plotly, a well-known graphing library.

This script will focus on data exploration, instead of a deep focus on
Plotly. If you'd like to review this custom script, you can find it in
our GitHub repository.

\hypertarget{first-inspection}{%
\subsection{First Inspection}\label{first-inspection}}

Let's reload the Titanic Dataset, and reacquaint ourselves with its
data:

    \begin{tcolorbox}[breakable, size=fbox, boxrule=1pt, pad at break*=1mm,colback=cellbackground, colframe=cellborder]
\prompt{In}{incolor}{4}{\boxspacing}
\begin{Verbatim}[commandchars=\\\{\}]
\PY{k+kn}{import} \PY{n+nn}{pandas} \PY{k}{as} \PY{n+nn}{pd}

\PY{c+c1}{\PYZsh{} Load data from our dataset file into a pandas dataframe}
\PY{o}{!}wget\PY{+w}{ }https://raw.githubusercontent.com/MicrosoftDocs/mslearn\PYZhy{}introduction\PYZhy{}to\PYZhy{}machine\PYZhy{}learning/main/Data/titanic.csv
\PY{o}{!}wget\PY{+w}{ }https://raw.githubusercontent.com/MicrosoftDocs/mslearn\PYZhy{}introduction\PYZhy{}to\PYZhy{}machine\PYZhy{}learning/main/graphing.py
\PY{n}{dataset} \PY{o}{=} \PY{n}{pd}\PY{o}{.}\PY{n}{read\PYZus{}csv}\PY{p}{(}\PY{l+s+s1}{\PYZsq{}}\PY{l+s+s1}{titanic.csv}\PY{l+s+s1}{\PYZsq{}}\PY{p}{,} \PY{n}{index\PYZus{}col}\PY{o}{=}\PY{k+kc}{False}\PY{p}{,} \PY{n}{sep}\PY{o}{=}\PY{l+s+s2}{\PYZdq{}}\PY{l+s+s2}{,}\PY{l+s+s2}{\PYZdq{}}\PY{p}{,} \PY{n}{header}\PY{o}{=}\PY{l+m+mi}{0}\PY{p}{)}

\PY{c+c1}{\PYZsh{} Let\PYZsq{}s take a look at the data}
\PY{n}{dataset}\PY{o}{.}\PY{n}{head}\PY{p}{(}\PY{p}{)}
\end{Verbatim}
\end{tcolorbox}

    \begin{Verbatim}[commandchars=\\\{\}]
--2023-08-19 03:19:06--
https://raw.githubusercontent.com/MicrosoftDocs/mslearn-introduction-to-machine-
learning/main/Data/titanic.csv
Resolving raw.githubusercontent.com (raw.githubusercontent.com){\ldots}
185.199.111.133, 185.199.110.133, 185.199.109.133, {\ldots}
Connecting to raw.githubusercontent.com
(raw.githubusercontent.com)|185.199.111.133|:443{\ldots} connected.
HTTP request sent, awaiting response{\ldots} 200 OK
Length: 61194 (60K) [text/plain]
Saving to: ‘titanic.csv.3’

titanic.csv.3       100\%[===================>]  59.76K  --.-KB/s    in 0.001s

2023-08-19 03:19:06 (91.5 MB/s) - ‘titanic.csv.3’ saved [61194/61194]

--2023-08-19 03:19:08--
https://raw.githubusercontent.com/MicrosoftDocs/mslearn-introduction-to-machine-
learning/main/graphing.py
Resolving raw.githubusercontent.com (raw.githubusercontent.com){\ldots}
185.199.111.133, 185.199.110.133, 185.199.109.133, {\ldots}
Connecting to raw.githubusercontent.com
(raw.githubusercontent.com)|185.199.111.133|:443{\ldots} connected.
HTTP request sent, awaiting response{\ldots} 200 OK
Length: 21511 (21K) [text/plain]
Saving to: ‘graphing.py.3’

graphing.py.3       100\%[===================>]  21.01K  --.-KB/s    in 0s

2023-08-19 03:19:08 (106 MB/s) - ‘graphing.py.3’ saved [21511/21511]

    \end{Verbatim}

            \begin{tcolorbox}[breakable, size=fbox, boxrule=.5pt, pad at break*=1mm, opacityfill=0]
\prompt{Out}{outcolor}{4}{\boxspacing}
\begin{Verbatim}[commandchars=\\\{\}]
   PassengerId  Survived  Pclass  \textbackslash{}
0            1         0       3
1            2         1       1
2            3         1       3
3            4         1       1
4            5         0       3

                                                Name     Sex   Age  SibSp  \textbackslash{}
0                            Braund, Mr. Owen Harris    male  22.0      1
1  Cumings, Mrs. John Bradley (Florence Briggs Th{\ldots}  female  38.0      1
2                             Heikkinen, Miss. Laina  female  26.0      0
3       Futrelle, Mrs. Jacques Heath (Lily May Peel)  female  35.0      1
4                           Allen, Mr. William Henry    male  35.0      0

   Parch            Ticket     Fare Cabin Embarked
0      0         A/5 21171   7.2500   NaN        S
1      0          PC 17599  71.2833   C85        C
2      0  STON/O2. 3101282   7.9250   NaN        S
3      0            113803  53.1000  C123        S
4      0            373450   8.0500   NaN        S
\end{Verbatim}
\end{tcolorbox}
        
    Take a careful look at the columns, and try to identify those columns
holding continuous, ordinal, categorical, or identity data.

We can display a brief summary of the \emph{dataypes} with panda's
\texttt{info()} method:

    \begin{tcolorbox}[breakable, size=fbox, boxrule=1pt, pad at break*=1mm,colback=cellbackground, colframe=cellborder]
\prompt{In}{incolor}{5}{\boxspacing}
\begin{Verbatim}[commandchars=\\\{\}]
\PY{n}{dataset}\PY{o}{.}\PY{n}{info}\PY{p}{(}\PY{p}{)}
\end{Verbatim}
\end{tcolorbox}

    \begin{Verbatim}[commandchars=\\\{\}]
<class 'pandas.core.frame.DataFrame'>
RangeIndex: 891 entries, 0 to 890
Data columns (total 12 columns):
 \#   Column       Non-Null Count  Dtype
---  ------       --------------  -----
 0   PassengerId  891 non-null    int64
 1   Survived     891 non-null    int64
 2   Pclass       891 non-null    int64
 3   Name         891 non-null    object
 4   Sex          891 non-null    object
 5   Age          714 non-null    float64
 6   SibSp        891 non-null    int64
 7   Parch        891 non-null    int64
 8   Ticket       891 non-null    object
 9   Fare         891 non-null    float64
 10  Cabin        204 non-null    object
 11  Embarked     889 non-null    object
dtypes: float64(2), int64(5), object(5)
memory usage: 83.7+ KB
    \end{Verbatim}

    We can see several columns stored as numerical data (the \texttt{int64}
or \texttt{float64} types), while others contain more complex data types
(those with \texttt{object} as Dtype)

\hypertarget{visualising-ordinal-data}{%
\subsection{Visualising Ordinal Data}\label{visualising-ordinal-data}}

Let's visualize some ordinal data. We have available:

\begin{enumerate}
\def\labelenumi{\arabic{enumi}.}
\tightlist
\item
  \texttt{Pclass} - the ticket class
\item
  \texttt{Parch} - the number of parents or children on the ship
\item
  \texttt{sibsp} - the number of siblings or spouses on the ship
\end{enumerate}

We can view ordinal data with almost any kind of graph. We'll start with
a simple histogram that describes relationships between the ticket class
and the likelihood of survival.

    \begin{tcolorbox}[breakable, size=fbox, boxrule=1pt, pad at break*=1mm,colback=cellbackground, colframe=cellborder]
\prompt{In}{incolor}{6}{\boxspacing}
\begin{Verbatim}[commandchars=\\\{\}]
\PY{k+kn}{import} \PY{n+nn}{graphing}

\PY{n}{graphing}\PY{o}{.}\PY{n}{histogram}\PY{p}{(}\PY{n}{dataset}\PY{p}{,} \PY{n}{label\PYZus{}x}\PY{o}{=}\PY{l+s+s1}{\PYZsq{}}\PY{l+s+s1}{Pclass}\PY{l+s+s1}{\PYZsq{}}\PY{p}{,} \PY{n}{label\PYZus{}y}\PY{o}{=}\PY{l+s+s1}{\PYZsq{}}\PY{l+s+s1}{Survived}\PY{l+s+s1}{\PYZsq{}}\PY{p}{,} \PY{n}{histfunc}\PY{o}{=}\PY{l+s+s1}{\PYZsq{}}\PY{l+s+s1}{avg}\PY{l+s+s1}{\PYZsq{}}\PY{p}{,} \PY{n}{include\PYZus{}boxplot}\PY{o}{=}\PY{k+kc}{True}\PY{p}{)}
\end{Verbatim}
\end{tcolorbox}

    
    
    The box and whisker plot (top) shows that at least half the people had
third-class tickets - note how the median and maximum of the plot both
sit at \texttt{Pclass\ =\ 3}.

The histogram shows that people in second and third class tended not to
survive the wreck.

Let's look at how survival varies, depending on whether a passenger had
parents or children on the ship

    \begin{tcolorbox}[breakable, size=fbox, boxrule=1pt, pad at break*=1mm,colback=cellbackground, colframe=cellborder]
\prompt{In}{incolor}{7}{\boxspacing}
\begin{Verbatim}[commandchars=\\\{\}]
\PY{n}{graphing}\PY{o}{.}\PY{n}{multiple\PYZus{}histogram}\PY{p}{(}\PY{n}{dataset}\PY{p}{,} 
                            \PY{n}{label\PYZus{}x}\PY{o}{=}\PY{l+s+s1}{\PYZsq{}}\PY{l+s+s1}{Pclass}\PY{l+s+s1}{\PYZsq{}}\PY{p}{,} \PY{c+c1}{\PYZsh{} group by ticket class}
                            \PY{n}{label\PYZus{}group}\PY{o}{=}\PY{l+s+s2}{\PYZdq{}}\PY{l+s+s2}{Parch}\PY{l+s+s2}{\PYZdq{}}\PY{p}{,} \PY{c+c1}{\PYZsh{} colour by no parents or children}
                            \PY{n}{label\PYZus{}y}\PY{o}{=}\PY{l+s+s1}{\PYZsq{}}\PY{l+s+s1}{Survived}\PY{l+s+s1}{\PYZsq{}}\PY{p}{,} 
                            \PY{n}{histfunc}\PY{o}{=}\PY{l+s+s2}{\PYZdq{}}\PY{l+s+s2}{avg}\PY{l+s+s2}{\PYZdq{}}\PY{p}{)}
\end{Verbatim}
\end{tcolorbox}

    
    
    For first and second class ticket holders, people in larger family
groups appear to have had better rates of survival. However, this
doesn't seem to be the case for third class passengers.

Lastly, let's see if those with different ticket types tended to be in
different sized families. For data spread analysis, a box and whisker is
a nice alternative to histograms.

    \begin{tcolorbox}[breakable, size=fbox, boxrule=1pt, pad at break*=1mm,colback=cellbackground, colframe=cellborder]
\prompt{In}{incolor}{8}{\boxspacing}
\begin{Verbatim}[commandchars=\\\{\}]
\PY{n}{graphing}\PY{o}{.}\PY{n}{box\PYZus{}and\PYZus{}whisker}\PY{p}{(}\PY{n}{dataset}\PY{p}{,} \PY{n}{label\PYZus{}x}\PY{o}{=}\PY{l+s+s2}{\PYZdq{}}\PY{l+s+s2}{Pclass}\PY{l+s+s2}{\PYZdq{}}\PY{p}{,} \PY{n}{label\PYZus{}y}\PY{o}{=}\PY{l+s+s2}{\PYZdq{}}\PY{l+s+s2}{SibSp}\PY{l+s+s2}{\PYZdq{}}\PY{p}{)}
\end{Verbatim}
\end{tcolorbox}

    
    
    Most values are zero. This shows that most people traveled without
siblings and without a partner. There are no obvious differences in this
value between the different ticket classes.

\hypertarget{visualising-continuous-data}{%
\subsection{Visualising Continuous
Data}\label{visualising-continuous-data}}

\emph{Continuous} data are usually best viewed using either:

\begin{enumerate}
\def\labelenumi{\arabic{enumi}.}
\tightlist
\item
  An XY scatter plot, especially for relationships between two
  continuous features
\item
  Histograms or Box and Whisker plots, to look at the spread of data
\end{enumerate}

Our dataset has \texttt{Age} and \texttt{Fare} as continuous data
columns. Let's view them:

    \begin{tcolorbox}[breakable, size=fbox, boxrule=1pt, pad at break*=1mm,colback=cellbackground, colframe=cellborder]
\prompt{In}{incolor}{9}{\boxspacing}
\begin{Verbatim}[commandchars=\\\{\}]
\PY{n}{graphing}\PY{o}{.}\PY{n}{scatter\PYZus{}2D}\PY{p}{(}\PY{n}{dataset}\PY{p}{,} \PY{n}{label\PYZus{}x}\PY{o}{=}\PY{l+s+s2}{\PYZdq{}}\PY{l+s+s2}{Age}\PY{l+s+s2}{\PYZdq{}}\PY{p}{,} \PY{n}{label\PYZus{}y}\PY{o}{=}\PY{l+s+s2}{\PYZdq{}}\PY{l+s+s2}{Fare}\PY{l+s+s2}{\PYZdq{}}\PY{p}{)}
\end{Verbatim}
\end{tcolorbox}

    
    
    We don't see an obvious relationship between \texttt{Age} and
\texttt{Fare}.

Does the cost of a fare, or the person's age, have any relationship with
likelihood of survival?

    \begin{tcolorbox}[breakable, size=fbox, boxrule=1pt, pad at break*=1mm,colback=cellbackground, colframe=cellborder]
\prompt{In}{incolor}{10}{\boxspacing}
\begin{Verbatim}[commandchars=\\\{\}]
\PY{c+c1}{\PYZsh{} Plot Fare vs Survival}
\PY{n}{graphing}\PY{o}{.}\PY{n}{histogram}\PY{p}{(}\PY{n}{dataset}\PY{p}{,} \PY{n}{label\PYZus{}x}\PY{o}{=}\PY{l+s+s2}{\PYZdq{}}\PY{l+s+s2}{Fare}\PY{l+s+s2}{\PYZdq{}}\PY{p}{,} \PY{n}{label\PYZus{}y}\PY{o}{=}\PY{l+s+s2}{\PYZdq{}}\PY{l+s+s2}{Survived}\PY{l+s+s2}{\PYZdq{}}\PY{p}{,} \PY{n}{histfunc}\PY{o}{=}\PY{l+s+s2}{\PYZdq{}}\PY{l+s+s2}{avg}\PY{l+s+s2}{\PYZdq{}}\PY{p}{,} \PY{n}{nbins}\PY{o}{=}\PY{l+m+mi}{30}\PY{p}{,} \PY{n}{title}\PY{o}{=}\PY{l+s+s2}{\PYZdq{}}\PY{l+s+s2}{Fare vs Survival}\PY{l+s+s2}{\PYZdq{}}\PY{p}{,} \PY{n}{include\PYZus{}boxplot}\PY{o}{=}\PY{k+kc}{True}\PY{p}{,} \PY{n}{show}\PY{o}{=}\PY{k+kc}{True}\PY{p}{)}

\PY{c+c1}{\PYZsh{} Plot Age vs Survival}
\PY{n}{graphing}\PY{o}{.}\PY{n}{histogram}\PY{p}{(}\PY{n}{dataset}\PY{p}{,} \PY{n}{label\PYZus{}x}\PY{o}{=}\PY{l+s+s2}{\PYZdq{}}\PY{l+s+s2}{Age}\PY{l+s+s2}{\PYZdq{}}\PY{p}{,} \PY{n}{label\PYZus{}y}\PY{o}{=}\PY{l+s+s2}{\PYZdq{}}\PY{l+s+s2}{Survived}\PY{l+s+s2}{\PYZdq{}}\PY{p}{,} \PY{n}{histfunc}\PY{o}{=}\PY{l+s+s2}{\PYZdq{}}\PY{l+s+s2}{avg}\PY{l+s+s2}{\PYZdq{}}\PY{p}{,} \PY{n}{title}\PY{o}{=}\PY{l+s+s2}{\PYZdq{}}\PY{l+s+s2}{Age vs Survival}\PY{l+s+s2}{\PYZdq{}}\PY{p}{,} \PY{n}{nbins}\PY{o}{=}\PY{l+m+mi}{30}\PY{p}{,} \PY{n}{include\PYZus{}boxplot}\PY{o}{=}\PY{k+kc}{True}\PY{p}{)}
\end{Verbatim}
\end{tcolorbox}

    
    
    
    
    The boxplot (top) of the first figure shows us that most people held
tickets that cost less than £25, and the histogram shows us that people
with more expensive tickets tended to survive.

Our second figure indicates passengers were about 30 years old on
average, and that most children under 10 years old survived, unlike most
adults.

\hypertarget{visualising-categorical-data}{%
\subsection{Visualising Categorical
Data}\label{visualising-categorical-data}}

Our Titanic dataset has the following \emph{categorical} columns: *
\texttt{Sex} (Male, Female) * \texttt{Embarked} - the port of
ambarkation (C, Q, or S) * \texttt{Cabin} (many options) *
\texttt{Survival} (0 = no, 1 = yes)

Categorical data are usually viewable in a similar way to ordinal data,
but with data viewed as order-less groups. Alternatively, categories
appear as colors, or groups, in other kinds of plots.

Plotting categorical data against other categorical data shows how data
is clustered. This is little more than a colored table. Let's do this
now:

    \begin{tcolorbox}[breakable, size=fbox, boxrule=1pt, pad at break*=1mm,colback=cellbackground, colframe=cellborder]
\prompt{In}{incolor}{11}{\boxspacing}
\begin{Verbatim}[commandchars=\\\{\}]
\PY{k+kn}{import} \PY{n+nn}{plotly}\PY{n+nn}{.}\PY{n+nn}{graph\PYZus{}objects} \PY{k}{as} \PY{n+nn}{go}
\PY{k+kn}{import} \PY{n+nn}{numpy} \PY{k}{as} \PY{n+nn}{np}

\PY{c+c1}{\PYZsh{} Create some simple functions}
\PY{c+c1}{\PYZsh{} Read their descriptions to find out more }
\PY{k}{def} \PY{n+nf}{get\PYZus{}rows}\PY{p}{(}\PY{n}{sex}\PY{p}{,} \PY{n}{port}\PY{p}{)}\PY{p}{:}
\PY{+w}{    }\PY{l+s+sd}{\PYZsq{}\PYZsq{}\PYZsq{}Returns rows that match in terms of sex and embarkment port\PYZsq{}\PYZsq{}\PYZsq{}}
    \PY{k}{return} \PY{n}{dataset}\PY{p}{[}\PY{p}{(}\PY{n}{dataset}\PY{o}{.}\PY{n}{Embarked} \PY{o}{==} \PY{n}{port}\PY{p}{)} \PY{o}{\PYZam{}} \PY{p}{(}\PY{n}{dataset}\PY{o}{.}\PY{n}{Sex} \PY{o}{==} \PY{n}{sex}\PY{p}{)}\PY{p}{]}

\PY{k}{def} \PY{n+nf}{proportion\PYZus{}survived}\PY{p}{(}\PY{n}{sex}\PY{p}{,} \PY{n}{port}\PY{p}{)}\PY{p}{:}
\PY{+w}{    }\PY{l+s+sd}{\PYZsq{}\PYZsq{}\PYZsq{}Returns the proportion of people meeting criteria who survived\PYZsq{}\PYZsq{}\PYZsq{}}
    \PY{n}{survived} \PY{o}{=} \PY{n}{get\PYZus{}rows}\PY{p}{(}\PY{n}{sex}\PY{p}{,} \PY{n}{port}\PY{p}{)}\PY{o}{.}\PY{n}{Survived}
    \PY{k}{return} \PY{n}{np}\PY{o}{.}\PY{n}{mean}\PY{p}{(}\PY{n}{survived}\PY{p}{)}

\PY{c+c1}{\PYZsh{} Make two columns of data \PYZhy{} together these represent each combination}
\PY{c+c1}{\PYZsh{} of sex and embarkment port}
\PY{n}{sexes} \PY{o}{=} \PY{p}{[}\PY{l+s+s2}{\PYZdq{}}\PY{l+s+s2}{male}\PY{l+s+s2}{\PYZdq{}}\PY{p}{,} \PY{l+s+s2}{\PYZdq{}}\PY{l+s+s2}{male}\PY{l+s+s2}{\PYZdq{}}\PY{p}{,} \PY{l+s+s2}{\PYZdq{}}\PY{l+s+s2}{male}\PY{l+s+s2}{\PYZdq{}}\PY{p}{,} \PY{l+s+s2}{\PYZdq{}}\PY{l+s+s2}{female}\PY{l+s+s2}{\PYZdq{}}\PY{p}{,}\PY{l+s+s2}{\PYZdq{}}\PY{l+s+s2}{female}\PY{l+s+s2}{\PYZdq{}}\PY{p}{,} \PY{l+s+s2}{\PYZdq{}}\PY{l+s+s2}{female}\PY{l+s+s2}{\PYZdq{}}\PY{p}{]}
\PY{n}{ports} \PY{o}{=} \PY{p}{[}\PY{l+s+s2}{\PYZdq{}}\PY{l+s+s2}{C}\PY{l+s+s2}{\PYZdq{}}\PY{p}{,} \PY{l+s+s2}{\PYZdq{}}\PY{l+s+s2}{Q}\PY{l+s+s2}{\PYZdq{}}\PY{p}{,} \PY{l+s+s2}{\PYZdq{}}\PY{l+s+s2}{S}\PY{l+s+s2}{\PYZdq{}} \PY{p}{]} \PY{o}{*} \PY{l+m+mi}{2}

\PY{c+c1}{\PYZsh{} Calculate the number of passengers at each port + sex combination}
\PY{n}{passenger\PYZus{}count} \PY{o}{=} \PY{p}{[}\PY{n+nb}{len}\PY{p}{(}\PY{n}{get\PYZus{}rows}\PY{p}{(}\PY{n}{sex}\PY{p}{,} \PY{n}{port}\PY{p}{)}\PY{p}{)} \PY{k}{for} \PY{n}{sex}\PY{p}{,}\PY{n}{port} \PY{o+ow}{in} \PY{n+nb}{zip}\PY{p}{(}\PY{n}{sexes}\PY{p}{,} \PY{n}{ports}\PY{p}{)}\PY{p}{]}

\PY{c+c1}{\PYZsh{} Calculate the proportion of passengers from each port + sex combination who survived}
\PY{n}{passenger\PYZus{}survival} \PY{o}{=} \PY{p}{[}\PY{n}{proportion\PYZus{}survived}\PY{p}{(}\PY{n}{sex}\PY{p}{,} \PY{n}{port}\PY{p}{)} \PY{k}{for} \PY{n}{sex}\PY{p}{,}\PY{n}{port} \PY{o+ow}{in} \PY{n+nb}{zip}\PY{p}{(}\PY{n}{sexes}\PY{p}{,} \PY{n}{ports}\PY{p}{)}\PY{p}{]}

\PY{c+c1}{\PYZsh{} Combine into a single data frame}
\PY{n}{table} \PY{o}{=} \PY{n}{pd}\PY{o}{.}\PY{n}{DataFrame}\PY{p}{(}\PY{n+nb}{dict}\PY{p}{(}
    \PY{n}{sex}\PY{o}{=}\PY{n}{sexes}\PY{p}{,}
    \PY{n}{port}\PY{o}{=}\PY{n}{ports}\PY{p}{,}
    \PY{n}{passenger\PYZus{}count}\PY{o}{=}\PY{n}{passenger\PYZus{}count}\PY{p}{,}
    \PY{n}{passenger\PYZus{}survival\PYZus{}rate}\PY{o}{=}\PY{n}{passenger\PYZus{}survival}
\PY{p}{)}\PY{p}{)}

\PY{c+c1}{\PYZsh{} Make a bubble plot}
\PY{c+c1}{\PYZsh{} This is just a scatter plot but each entry in the plot }
\PY{c+c1}{\PYZsh{} has a size and colour. We set colour to passenger\PYZus{}survival }
\PY{c+c1}{\PYZsh{} and size to the number of passengers }
\PY{n}{graphing}\PY{o}{.}\PY{n}{scatter\PYZus{}2D}\PY{p}{(}\PY{n}{table}\PY{p}{,} 
                    \PY{n}{label\PYZus{}colour}\PY{o}{=}\PY{l+s+s2}{\PYZdq{}}\PY{l+s+s2}{passenger\PYZus{}survival\PYZus{}rate}\PY{l+s+s2}{\PYZdq{}}\PY{p}{,} 
                    \PY{n}{label\PYZus{}size}\PY{o}{=}\PY{l+s+s2}{\PYZdq{}}\PY{l+s+s2}{passenger\PYZus{}count}\PY{l+s+s2}{\PYZdq{}}\PY{p}{,} 
                    \PY{n}{size\PYZus{}multiplier}\PY{o}{=}\PY{l+m+mf}{0.3}\PY{p}{,}
                    \PY{n}{title}\PY{o}{=}\PY{l+s+s2}{\PYZdq{}}\PY{l+s+s2}{Bubble Plot of Categorical Data}\PY{l+s+s2}{\PYZdq{}}\PY{p}{)}
\end{Verbatim}
\end{tcolorbox}

    
    
    It appears that women have a much higher survival rate than men, but
there were more men on the ship.

We can also see that most people boarded at Port \texttt{S}
(``Southampton''). It does seem that there is a weak relationship
between the port of boarding and survival.

\hypertarget{summary}{%
\subsection{Summary}\label{summary}}

You've learned about different types of data, and you practiced data
exploration through graphs.

Through these, we've discovered that some features are related to
others, and that many features seem to influence the survival rate.


    % Add a bibliography block to the postdoc
    
    
    
\end{document}
